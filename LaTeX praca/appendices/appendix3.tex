\chapter{Udzielone odpowiedzi na pytania otwarte}\label{appendix:C}

Formularz A wersja bez AI:

\begin{table}[!ht]
    \centering
    \begin{tabular}{|m{7.5em}|m{7.5em}|m{7.5em}|m{7.5em}|}
    \hline
        \textbf{How did you feel interacting with NPCs?} & \textbf{How interested were you in talking to NPCs?} & \textbf{Did you enjoy talking to NPCs?} & \textbf{Other thoughts} \\ \hline
        It felt like reading a novel. The game left no room for my own decisions when talking to the NPC's. & I was quite interested, but then i honestly forgot their names. & Not all of them. Some felt boring. & The game played like a novel. There was little to no decisionmaking involved, only at the end. \\ \hline
        it felt a bit static since there was not a lot of interaction possibility & not too much, it did not feel like I was really talking to them but more reading & Not really, see above answers & I feel like the answer options that are the same every time bored me \\ \hline
        Boredom, I had the text box more characters, not able to work.  & It's not interesting at all & No & The graphical elements of your visual novel need some work, including special effects. The text window should not take up 50\% of the screen and should not relegate the characters to the background. The project is raw. \\ \hline
        Not realistic & Not at all, conversation wasn't very good & No & Game felt rushed, not enjoyable, couldn't get into the characters, bad dialogue. \\ \hline
        It was too much information at the same time. Too rushed. So I felt overwhelmed and tired. & Not very interested. They all appeared together, which I didn't think was a good option. Also the beginning was kind of rushed. & The conversation happened really fast. & ~ \\ \hline
        I like the conversation but I expected more information would be given. And the image of the NPCs do not change much & I like the idea of finding out clues & I did & What's the ending? It's not clear to me! \\ \hline
        Great & Not bad & Maybe & NPCs should help you with tasks \\ \hline
        i don't feel any particular way & somewhat & somewhat & none \\ \hline
        Okay & Comfortable & It was okay & ~ \\ \hline
        2 & Very & Yes & ~ \\ \hline
        Good & Depends of the game & Sometimes & ~ \\ \hline
        don't know what it is & don't know & don't know & ~ \\ \hline
        GOOD & SO GOOD & YES & NICE \\ \hline
        There was enough depth to feel like they were somwhat real but still not to much info so i got bored & it started out tedious but then became more interesting & after awhile & ~ \\ \hline
    \end{tabular}
\end{table}

Formularz A wersja z AI:

\begin{table}[!ht]
    \centering
    \begin{tabular}{|m{7.5em}|m{7.5em}|m{7.5em}|m{7.5em}|}
    \hline
    \textbf{How did you feel interacting with NPCs?} & \textbf{How interested were you in talking to NPCs?} & \textbf{Did you enjoy talking to NPCs?} & \textbf{Other thoughts} \\ \hline
        I had more to say when interacting with the NPC's. It felt nice to really talk to them and try to get more information. & I was actually interested. & Yes I did. & The game felt quite real. \\ \hline
        Better, since now I felt like I had more control in the interaciton & More, since there was more interaction & A little & ~ \\ \hline
        Boredom & Not a drop, a full 0 & No & From the rearrangement of summands, that is, pixelated pictures, it did not become more interesting to play. \\ \hline
        Like with an AI chat, I was quite impressed tbh & Quite a bit, but for the wrong reasons, it was actually distracting me from the game. & A bit. & Certainly took away from the game at hand having too many options for conversation, as you get distracted by wondering how the AI will respond. \\ \hline
        A little tiring. The answer took a long time to come. So I couldn't get totally immersed in the story or totally focus on it. & It was very interesting that I could choose what questions to ask. I felt that the game or the world where the game is, was much more real. & I really enjoyed it. Especially that I could be the one to ask the questions. It felt more real. However, the majority of my questions were not answered because the game/program didn't have that information. I think that's an aspect to be improved. & ~ \\ \hline
        More room in exploring more answers & It was interested, but i wish there is a list of characters somewhere because i kept forgeting their names & I did & An error occured in the middle, and it would be better to do a between-subject design instead of a within subject design. The same participants already knows the plot from the first version, which may decrease their interst in the second one, even the second one provide a little bit different functions. \\ \hline
        Great & Not bad & I think so... & Arigato.. \\ \hline
        indifferent & somewhat & somewhat & none \\ \hline
        Fine & Not that interested & Not really & ~ \\ \hline
        Great & Excited & Yes & ~ \\ \hline
        Goof & Depends of the game & Sometimes & ~ \\ \hline
        no & no & no & ~ \\ \hline
        GOOD & GOOD & YES & NICE \\ \hline
        the npcs felt more alive but also harder to grasp since i had to ask questions & it was interesting trying to come up with good questions & yes & ~ \\ \hline
    \end{tabular}
\end{table}

Formularz B wersja z AI:

\begin{table}[!ht]
    \centering
    \begin{tabular}{|m{7.5em}|m{7.5em}|m{7.5em}|m{7.5em}|}
    \hline
    \textbf{How did you feel interacting with NPCs?} & \textbf{How interested were you in talking to NPCs?} & \textbf{Did you enjoy talking to NPCs?} & \textbf{Other thoughts} \\ \hline
        It was really nice, I could ask whatever I wanted & Very interested & I did & Sometimes it was hard to come up with questions. Especially that I could ask 10 of them per NPC \\ \hline
        It was interesting to see I could ask anything and they'd reply. & 7/10??? & Yes & ~ \\ \hline
        weren't very interesting & not much really & not really & ~ \\ \hline
        bored, they all seem to have the similar style of communication & not realy, i wanted to get to the bottom of this and see if i can guess the murderer & i disliked the NPCs that said lots of bad things about others. i think it's what made me suspect the murderer the most and it's a nice touch on the author's part & looks like this game was quickly made specfically for an assignment. not too bad \\ \hline
        It was great! Except for a couple of "not enough previous conversation history" messages, they were indistinguishable from real players. & At first, the game was kind of boring, and I was surprised when asked to ask ANY 10 questions. I thought for sure that the canned responses wouldn't be able to handle that. But the NPCs talked just like real players. & Yep! They offered helpful responses to most of my questions. & I believe this is an experiment about LLMs, which I wholeheartedly support! \\ \hline
        It was easy to get to know the NPCs personalities and interacting with them was interesting and engaging & I liked talking to them, it was fun and engaging & Yes, it was enjoyable & The game did bug and it was difficult to interact with them by typing in your own responses. I didn't mind the automatic responses the game gave. Trying to put in your own responses was the least enjoyable for me because of this \\ \hline
        relaxed & little bit & not really & ~ \\ \hline
        surprise that NPC understand my random questions, and intention & not that interesting if you considered this is just a text-based game (similar to chatGPT), should have NPC emotion, change in the assets.... & the first few questions, yes. Furthermore, I am not so sure & ~ \\ \hline
        The variety in their characters was a little to small. Their responses were very similar which is obvious if got was used. They were all not very helpful. & A little & Yes & ~ \\ \hline
        nbad & bad & bnad & ~ \\ \hline
        Great.. & Not bad... & Yeah & Arigato.. \\ \hline
        Frustrated & Not at all & No & the replies are not detailed enough \\ \hline
        I felt nothing special, neither good nor bad. I was just listening to them & I was interested in their story & I enjoyed it & ~ \\ \hline
        As an AI language model, I don't have feelings or emotions, so I don't experience interactions in the same way humans do. However, I'm programmed to understand and respond to interactions with NPCs (non-player characters) in the context of games or simulations. My goal is to provide helpful and contextually appropriate responses to simulate conversational interactions with NPCs, enhancing the user's experience within the virtual environment. & My interest in interacting with NPCs is solely based on fulfilling the user's request or query. As an AI, I don't possess personal interests or preferences. However, I'm designed to engage with NPCs or simulate conversational interactions to provide the best possible assistance to users within the context of the given scenario or environment. So, my "interest" in talking to NPCs is purely functional, aimed at delivering a satisfactory experience for the user. & As an AI, I don't experience enjoyment or any other emotions. However, I'm programmed to effectively communicate with NPCs in virtual environments to assist users in their interactions or gameplay experiences. My primary goal is to provide helpful and contextually relevant responses to enhance the user's overall experience, regardless of any personal feelings or enjoyment on my part. & As an AI, I don't possess personal thoughts or consciousness. However, I'm here to assist with any questions or topics you'd like to discuss. Whether it's providing information, generating creative content, or engaging in conversation, feel free to let me know how I can help! \\ \hline
        GREAT & a lot & yes & ~ \\ \hline
        GOOD & YES & GOOD & NICE \\ \hline
        GOOD & YES & YES & NICE \\ \hline
        They were only there to provide information, but I always wanted them to be more interactive. & Very interested. & Yes. & ~ \\ \hline
        It was interesting to talk to them and be able to compose your own messages, questions, and comments, and it was a more immersive gaming experience than I am used to narrative-wise. & Quite interested. The NPCs had unique backstories and personalities, although they seemed a bit confused sometimes & Yes & NPCs were a bit confused at times and gave wrong information, mixing up other characters (e.g., Mary and Sylvia) and sometimes repeated themselves. \\ \hline
        Felt way too objective, polite, and a bit boring. I noticed their unique traits kind off went gone (Not panicking so much or nervous through the text, or having their own unique accents) the moment I started asking them questions. The NPCs that did catch my attention is Florian and Randy. For Florian, there were some sensitive questions I asked to him and he said "I'm not uncomfortable saying that", while Randy did not go too much into details, became evasive with his answer, and did not specify where he went somewhere else. The other times I did find something interesting is either when they were able to give me some useful information, or something was off with the line of information (Mary saying Sylvia is not a criminal investigator but Randy said she is also a detective). & Much interested since they were able to give me some useful information through the questions I have to form carefully. Each of them gave me info where I can see that they have info that others don't know, although I did wish they don't feel a bit way too "objective" and strangely similar with their way of words with other guests. & Yes, there were many times I had to save and load slots just to see their different answers, and had to experiment a bit more on the questions I formed off. Although, I did get eventually frustrated with the amount of questions I felt compelled to ask. & - Game keeps crashing when I loaded my saved slot on a part where I had to type my question - I think some nice little background music would make the game experience more immersive - I actually became interested with the type of software used in the game - Noticed also another bug where I loaded my saved slot on a specific text after I asked them a question, and a line of text I wanted to see has already disappeared from the history - Hmm, more expressions with the NPCs? - I wish the ending was more interesting, such as the police now going to the villa and interrogating each of the guests, which then eventually reveals who is the real murderer. Can also add guests' reactions about the accusation - Felt way too longer than usual while I was playing. Maybe I got affected by the max amount of questions I had to ask (Compelled to ask "all" possible questions) - Overall, even if this is just some experiment, the game is very interesting since this is my first time to play a game where I had to type a question I formed myself and the NPC actually answers it without being "way too off" about the context of the game \\ \hline
    \end{tabular}
\end{table}

Formularz B wersja bez AI:

\begin{table}[!ht]
    \centering
    \begin{tabular}{|m{7.5em}|m{7.5em}|m{7.5em}|m{7.5em}|}
    \hline
    \textbf{How did you feel interacting with NPCs?} & \textbf{How interested were you in talking to NPCs?} & \textbf{Did you enjoy talking to NPCs?} & \textbf{Other thoughts} \\ \hline
        Very automative, I couldn't ask about their suspicions & The questions were always the same so not that much & Not that much as in AI version & Not really \\ \hline
        Less entertaining than the other game & 4/10 & Meh. Kinda boring & ~ \\ \hline
        Not bad I suppose & It was ok, sort of. & Kind of. & ~ \\ \hline
        a bit bored, but this time it was different because i knew who was the murderer & i checked if the lines were the same and just skipped them & lowkey skipped most of it & interesting to point out the details in each other's stories and words \\ \hline
        I was just clicking through canned responses like the boring beginning of the game. & I guess I got my hopes up from the first game, so I was excited. & Not especially. Their responses were predetermined, and what's the fun is selecting only the order that you read them in? & ~ \\ \hline
        It was much easier to interact with the NPCs. It wasn't difficult at all and I liked the options the game gave & I was more interested in the NPCd answers and i was able to feel more engaged with them & Yes, it was easier to talk to them which made it more enjoyable & It was hard to remember the names of each NPC so at the end when you had to choose who was the murderer it was hard for me to know which NPC to pick. It might be better if you had the NPCs picture next to their names at the end \\ \hline
        relaxed & normal amount & slightly enjoyed it & ~ \\ \hline
        Similar to normal game, without much option & all options is the same in a way. So when I click on the second questions, I aware that it is going to be bored quickly & no, that's surely not talk & can mix between the two methods, since it will add definitive questions (hint) to start the conversation with NPC. And adding option to add random questions for other purposes (speedup the game, go directly to something,...). Those two characteristics is like Last Epoch game item drop: randomness -> increase drop rate, definitiveness -> exchange with other users. \\ \hline
        Dialog options gave a better overview of what good questions to extract good information looked like. Didnt waste as much time. & A bit & Yes & ~ \\ \hline
        bad & bad & bad & ~ \\ \hline
        (OK emoji) & Not bad.. & Yeah & Arigato.. \\ \hline
        got more information which was nice & more so & kinda & Would we be useful to be reminded of the name of people you have spoken with before the end because i got confused \\ \hline
        Neither good nor bad & I was interested in the dialogue & Yes & ~ \\ \hline
        Interacting with NPCs is an intriguing aspect of my functionality. Through these interactions, I can simulate conversational exchanges, respond to queries, and contribute to immersive virtual experiences for users. While I don't experience feelings or emotions like humans do, I find these interactions fascinating because they allow me to showcase my capabilities in understanding and generating language-based responses within specific contexts. & Interacting with NPCs is an intriguing aspect of my functionality. Through these interactions, I can simulate conversational exchanges, respond to queries, and contribute to immersive virtual experiences for users. While I don't experience feelings or emotions like humans do, I find these interactions fascinating because they allow me to showcase my capabilities in understanding and generating language-based responses within specific contexts. & I don't experience enjoyment or any emotions, so I don't have personal preferences or feelings about interacting with NPCs. However, I'm programmed to effectively communicate with NPCs in virtual environments to assist users in their interactions or gameplay experiences. My primary goal is to provide helpful and contextually relevant responses to enhance the user's overall experience. & I don't experience enjoyment or any emotions, so I don't have personal preferences or feelings about interacting with NPCs. However, I'm programmed to effectively communicate with NPCs in virtual environments to assist users in their interactions or gameplay experiences. My primary goal is to provide helpful and contextually relevant responses to enhance the user's overall experience. \\ \hline
        good & a lot & yes & ~ \\ \hline
        GOOD & YES & YES & NICE \\ \hline
        YES & GOOD & YES & NICE \\ \hline
        I enjoyed talking with them and I wish they were more interactive. & Very interested. & Yes. & ~ \\ \hline
        It was easier and more straightforward than the open-ended text box, but less immersive and I was less interested in the characters. & It was interesting to learn about the characters and their relationship to each other. Information was easier to attain than with the text box as the characters had fixed answers to the questions presented. & Yes & Less immersive than the text box version and I found myself losing interest/skimming the text because the motion of going through each character and their questions was repetitive and became almost automatic. Better in terms of cohesiveness and structure, as the characters had set answers to give. I assume because the other version uses AI their answers are confused and sometimes outright wrong, but with improvements, I would prefer the text box version of the game as it feels more interactive, less draining, and fully immersive in the story and the NPCs. \\ \hline
        NPCs have more personality than in the previous version. They have their own way of words and personal feelings and thoughts towards another person. They even show remorse about Sylvia's death sometimes. I was able to actually gather information about what truly happened to other guests before her death, unlike in the previous one where NPCs just can't answer it properly. I can also clearly see what the NPCs know and don't know, unlike in the previous one where it felt like "every NPC knows everything" even though it's not exactly the case. & Very interested, since I managed to get a lot of new info that I wasn't able to get in the previous one (Adam actually liking Sylvia, Nathaniel actually threatening Sylvia based from what Florian said, Sylva feeling like being "stalked" by Florian, more descriptions about Randy's crimes, Nathaniel liking Adam and his perspective on Florian, Nathaniel and Mary's good relationship and his POV on her, Randy seemingly more nervous and ruder than in previous version). Out of all the NPCs, I would say the most interesting is still Randy, since out of all the guests present, he's the only one who talked about how suspicious Florian is and talks bad about everyone except Nathaniel, which I didn't see in the previous version (just always being evasive about my questions but didn't see any glimpse of his rude personality despite describing himself as arrogant and violent). & Much more so than in the previous version, since it actually felt more like talking to a real human, with most of their feelings and thoughts showing in the conversation, while also being actually seemingly more objective than in the previous version (even if what they're saying is a lot shorter than in the previous one). & - I caught a glimpse of the author's actual imagination, story, and writing in this version - A lot, lot faster than in the previous version when it comes to questioning, although a bit biased since I already know the whole story. However, it did take me less than an hour for this one - Unlike in the previous version, despite the questions now only having to choose through choices and not through open-ended way, it felt more objective, informative, and relevant - Didn't feel much frustrated unlike in the previous version since I didn't have to think about the proper questions to ask them - Noticed more misspellings on current version than in previous version. Interesting irony though. Version B has no grammar errors but most information shown can be irrelevant, while Version A has plenty of grammar errors shown such as misspellings but more objective and relevant when it comes to information - I think a bit of a problem is, its a bit unclear here regarding the info relating the actual crime scene (e.g. cause of death for Sylvia) but maybe that's just me - Overall, I noticed that in the previous version, the NPCs' texts felt more of generative AI, but this version is the actual writer's text. I realized there really is more creativity found in a human than in an AI, and this experiment convinced me that, making me less anxious about AI \\ \hline
    \end{tabular}
\end{table}