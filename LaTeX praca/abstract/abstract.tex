\abstract{
    % Polskie streszczenie
    W niniejszej pracy zbadano potencjał wykorzystania dużych modeli językowych (\gls{llm}) do poprawy jakości narracji
    i zaangażowania graczy w grach typu "visual novel". Wykorzystano system generujący dialogi na podstawie \gls{llm}, który
    następnie zaimplementowano w prototypowej grze. Przeprowadzono badania z udziałem graczy, w których oceniano
    imersję narracyjną, zaangażowanie oraz ogólną satysfakcję z gry w wersji z dialogami generowanymi przez \gls{llm} oraz
    wersji z wcześniej zdefiniowanymi dialogami. Wyniki badań wykazały, że uczestnicy odczuwali większą satysfakcję
    oraz zainteresowanie podczas interakcji z \gls{npc} zarządzanymi przez \gls{llm}. Potwierdziło to tezę, że wykorzystanie
    dużych modeli językowych może skutecznie podnieść jakość narracji w grach
    poprzez dostarczanie bardziej zindywidualizowanych i responsywnych doświadczeń dla graczy. Praca omawia również
    potencjalne dalsze kierunki badań i rozwoju systemów opartych na \gls{llm} w kontekście gier wideo.
}{
    % English abstract
    This work investigated the potential of utilizing large language models (\gls{llm}s) to enhance narrative quality
    and player engagement in visual novel-style games. A system for generating dialogues based on \gls{llm}s was used
    in a prototype game. Player studies were conducted to evaluate narrative immersion, engagement,
    and overall game satisfaction in both the \gls{llm}-generated dialogue version and a version with predefined dialogues.
    The results showed that participants experienced greater satisfaction and interest in interacting with the
    \gls{llm}-operated \gls{npc}s. This confirmed the
    hypothesis that leveraging \gls{llm}s can effectively improve narrative quality in games by providing more
    individualized and responsive experiences for players. The work also discusses potential further research
    directions and the development of \gls{llm}-based systems in the context of video games.
}