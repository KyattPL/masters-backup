\abstract{
    % Polskie streszczenie
    W niniejszej pracy zbadano potencjał wykorzystania dużych modeli językowych (LLM) do poprawy jakości narracji
    i zaangażowania graczy w grach typu "visual novel". Opracowano system generujący dialogi na podstawie LLM, który
    następnie zaimplementowano w prototypowej grze. Przeprowadzono badania z udziałem graczy, w których oceniano
    imersję narracyjną, zaangażowanie oraz ogólną satysfakcję z gry w wersji z dialogami generowanymi przez LLM oraz
    wersji z wcześniej zdefiniowanymi dialogami. Wyniki badań wykazały, że uczestnicy odczuwali większą imersję i
    zaangażowanie w wersji z dialogami generowanymi przez LLM, postrzegając je jako bardziej spójne i dopasowane do
    ich wyborów w grze. Potwierdziło to tezę, że wykorzystanie LLM może skutecznie podnieść jakość narracji w grach
    poprzez dostarczanie bardziej zindywidualizowanych i responsywnych doświadczeń dla graczy. Praca omawia również
    potencjalne dalsze kierunki badań i rozwoju systemów opartych na LLM w kontekście gier wideo.
}{
    % English abstract
    This work investigated the potential of utilizing large language models (LLMs) to enhance narrative quality
    and player engagement in visual novel-style games. A system for generating dialogues based on LLMs was developed
    and implemented in a prototype game. Player studies were conducted to evaluate narrative immersion, engagement,
    and overall game satisfaction in both the LLM-generated dialogue version and a version with predefined dialogues.
    The results showed that participants experienced greater immersion and engagement with the LLM-generated
    dialogues, perceiving them as more coherent and tailored to their in-game choices. This confirmed the
    hypothesis that leveraging LLMs can effectively improve narrative quality in games by providing more
    individualized and responsive experiences for players. The work also discusses potential further research
    directions and the development of LLM-based systems in the context of video games.
}