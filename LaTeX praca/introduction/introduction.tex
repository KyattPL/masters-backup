% !TEX encoding = UTF-8 Unicode 
% !TEX root = praca.tex

\chapter*{Wstęp}\label{chapter:introduction}

TODO

\section*{Zakres pracy}

TODO

\section*{Cel pracy}

Celem tej pracy jest zbadanie, w jaki sposób włączenie dużych modeli językowych (LLM) do gry typu „visual novel"
może zwiększyć imersję narracyjną i zaangażowanie gracza. Główną tezą jest to, że interaktywne dialogi generowane
przez LLM zapewnią bardziej spójną i dostosowaną do gracza narrację w porównaniu z wcześniej zdefiniowanymi
dialogami, co przełoży się na większe zaangażowanie i satysfakcję z gry.

\section*{Struktura pracy}

Praca podzielona jest na cztery rozdziały, podsumowanie oraz wnioski. Pierwszy rozdział przedstawia kontekst i
znaczenie narracji w grach wideo oraz omawia potencjalne korzyści wynikające z wykorzystania LLM. Drugi rozdział
przegląda istniejące prace związane z generowaniem narracji i dialogów w grach. Trzeci rozdział opisuje szczegóły
implementacji zaproponowanego systemu. W czwartym rozdziale przedstawione są wyniki badań z udziałem graczy oraz
analiza skuteczności rozwiązania. Praca kończy się podsumowaniem, wnioskami oraz wskazaniem możliwych kierunków
dalszych badań.