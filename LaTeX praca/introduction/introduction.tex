% !TEX encoding = UTF-8 Unicode 
% !TEX root = praca.tex

\chapter*{Wstęp}\label{chapter:introduction}

Narracja od zawsze stanowiła istotny element gier komputerowych, pozwalając graczom zanurzyć się w świecie
stworzonym przez projektantów. W miarę rozwoju technologii, możliwości opowiadania historii w grach stale
ewoluowały, poczynając od prostych tekstów, aż po złożone sekwencje filmowe i systemy dialogowe. Jednak tradycyjne
metody generowania narracji posiadają pewne ograniczenia, takie jak liniowość czy konieczność ręcznego tworzenia
każdego wątku fabularnego.

Z drugiej strony, postępy w dziedzinie sztucznej inteligencji, a w szczególności duże modele językowe (\gls{llm}),
oferują nowe możliwości w zakresie generowania dynamicznej, interaktywnej narracji w czasie rzeczywistym. Dzięki
zdolności do przetwarzania i generowania tekstu na poziomie ludzkim, \gls{llm} mogą tworzyć spójne i angażujące dialogi,
reagując na działania gracza i dostosowując fabułę do jego preferencji.

\section*{Zakres pracy}

W ramach niniejszej pracy zdecydowano się wykorzystać gotowe rozwiązanie z platformy Inworld \gls{ai} do generowania
dialogów z wykorzystaniem dużych modeli językowych. Ograniczono się do autorskiego prototypu gry zrobionego w
bibliotece \textit{Ren'Py}, w którym rozgrywka powinna zająć maksymalnie 30 minut. W celu przeprowadzenia eksperymentu
i porównania zaangażowania graczy przygotowano dwa formularze: A i B. W formularzu A uczestnicy najpierw grają
w wersję bez wspomagania \gls{ai}, a następnie w wersję z \gls{ai}, natomiast w formularzu B kolejność jest odwrotna.
Takie podejście pozwala na dokonanie porównania na ograniczonej liczbie uczestników.

\section*{Cel pracy}

Celem tej pracy jest zbadanie, w jaki sposób włączenie dużych modeli językowych (\gls{llm}) do gry typu „visual novel"
może zwiększyć imersję narracyjną i zaangażowanie gracza. Główną tezą jest to, że interaktywne dialogi generowane
przez \gls{llm} zapewnią bardziej spójną i dostosowaną do gracza narrację w porównaniu z wcześniej zdefiniowanymi
dialogami, co przełoży się na większe zaangażowanie i satysfakcję z gry.

\section*{Struktura pracy}

Praca podzielona została na 7 rozdziałów. W podsumowaniu znaleźć można końcowe wnioski z analizy danych oraz
potencjalne kroki do podjęcia w przyszłości w ramach tego obszaru.

W rozdziale \ref{chapter:ch1} przedstawiona jest historia narracji w grach komputerowych, obejmująca definicję narracji, jej
przedstawienie na przestrzeni lat oraz prześledzenie rozwoju na przykładzie serii "Final Fantasy". Rozdział \ref{chapter:ch2}
opisuje różne rodzaje narracji występujące w grach, z podziałem na struktury narracyjne (liniowa, łańcuch pereł,
rozgałęziająca się, park rozrywki, cegiełki) oraz sposoby przedstawiania narracji (cut scenki, tekst, dialogi
z \gls{npc}, poprzez świat gry).

Rozdział \ref{chapter:ch3} koncentruje się na systemach dialogowych, omawiając popularne ich współcześnie rodzaje a także
przedstawiając system poleceń w interaktywnej fikcji. W rozdziale \ref{chapter:ch4} opisane zostały dotychczasowe sposoby
generowania narracji z wykorzystaniem algorytmów sztucznej inteligencji oraz dużych modeli językowych.

Rozdział \ref{chapter:ch5} dotyczy zaangażowania gracza, definiując różne rodzaje zaangażowania oraz sposoby jego pomiaru. Kluczowy
rozdział \ref{chapter:ch6} przedstawia planowany eksperyment, w tym projekt gry, opis generatywnych agentów oraz zaplanowany
przebieg eksperymentu. Następnie w rozdziale \ref{chapter:ch7} zaprezentowane są wyniki eksperymentu, w tym demografia uczestników
oraz analiza danych. Na końcu znajdują się podsumowanie pracy, spisy, dodatki oraz bibliografia.