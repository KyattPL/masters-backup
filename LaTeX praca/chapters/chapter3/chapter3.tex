\graphicspath{{chapters/chapter3/imgs/}}

\chapter{Zaangażowanie gracza}\label{chapter:ch3}

By móc określić wpływ rozwiązań sztucznej inteligencji na angażującą narrację wewnątrz gry należy
zdefiniować w jaki sposób określić i mierzyć można zaangażowanie gracza. W tej sekcji podjęta
zostanie próba skategoryzowania zaangażowania i wybrania najodpowiedniejszej metody do wykorzystania
w ramach eksperymentu.

\section{Definicje i rodzaje zaangażowania}\label{section:ch3_1}

Pomiar i ocena zaangażowania graczy w gry wideo jest niezwykle skomplikowanym zadaniem.
Podstawową przeszkodą jest to, że zaangażowanie nie jest zjawiskiem jednowymiarowym\cite{eng_in_games}, łatwym do
uchwycenia\cite{measuring_user_exp}. Stanowi raczej złożony, subiektywny stan umysłu, na który składają się różne komponenty
takie jak: immersja, obecność, przepływ (flow), absorpcja psychologiczna czy dyssocjacja. Brak jest
powszechnie akceptowanej definicji i sposobu operacjonalizacji tego konstruktu\cite{measuring_user_exp}.

Kolejnym kluczowym utrudnieniem jest fakt, iż doświadczenie zaangażowania oparte jest na
nieuświadomionych, podświadomych procesach poznawczych i emocjonalnych, do których trudno uzyskać
introspekcyjny dostęp\cite{measuring_user_exp}. Wymaganie od gracza, by w trakcie rozgrywki analizował i werbalizował swoje
przeżycia, nieuchronnie zakłóca i niszczy naturalny przebieg tego delikatnego stanu umysłu.
Zaangażowanie jest więc zjawiskiem "ulotnym" i efemerycznym, które rozpada się pod wpływem przymusowej
uważności\cite{measuring_user_exp}.

Problemy pojawiają się również przy próbach retrospektywnego opisu i oceny zaangażowania po zakończeniu
sesji gry. Ludziom brakuje wspólnego, wyspecjalizowanego słownika, który pozwoliłby precyzyjnie
zwerbalizować subtelności i niuanse tego złożonego doświadczenia\cite{measuring_user_exp}.
Ograniczeni jesteśmy do używania bardzo ogólnych, niedookreślonych terminów takich jak "frajda",
"zaangażowanie" czy "zaabsorbowanie", które nijak nie oddają bogactwa i głębi przeżywanych
stanów\cite{measuring_user_exp}.

Co więcej, zaangażowanie jest zjawiskiem silnie ukontekstowionym, zanurzonym w wielu wzajemnie
przenikających się warstwach - samej rozgrywki, interakcji społecznych oraz fizycznego środowiska, w
którym odbywa się gra. Wszelkie próby jego wyizolowania i uproszczenia do czysto indywidualnego
doświadczenia gruntownie zniekształcają i wypaczają jego naturę.

W badaniach nad zaangażowaniem istotne są prace, które pomogły zdefiniować, gdzie ono rezyduje w
odniesieniu do szerokiego spektrum wymiarów lub stanów poznawczych i afektywnych, takich jak motywacja
i poczucie własnej skuteczności. Wskazują one, że teoria samookreślenia pomaga wyjaśnić, w jaki
sposób motywacja wewnętrzna - napędzana czynnikami takimi jak potrzeba kompetencji - jest ważnym
prekursorem zaangażowania. Zaangażowanie jest jednak konstruktem odrębnym od motywacji\cite{measuring_engagement}. Można je
postrzegać jako serię czasowych interakcji podczas wykonywania zadania, podczas gdy motywacja to
bardziej globalna osobista orientacja wobec nauki/zadania. Interakcja między nimi może utworzyć pętlę
sprzężenia zwrotnego, w której doświadczenie z zadaniem kształtuje bardziej zbliżone do stanu elementy
poczucia własnej skuteczności i motywacji, które z kolei wpływają na chęć ponownego zaangażowania się
użytkownika w zadanie\cite{measuring_engagement}.

Przedstawione zostaną najczęściej występujące w literaturze terminy pokrewne lub stanowiące część
zaangażowania:

\begin{itemize}
    \item Imersja (ang. \textit{Immersion})
    \item Obecność (ang. \textit{Presence})
    \item Przepływ / trans  (ang. \textit{Flow})
    \item Absorpcja (ang. \textit{Absorption})
    \item Dysocjacja (ang. \textit{Dissociation})
\end{itemize}

\begin{description}
    \item[Imersja] (zanurzenie) w grach wideo jest szeroko dyskutowanym pojęciem w pracach
          naukowych. Imersja zazwyczaj opisuje doświadczenie zaangażowania się w grę, przy zachowaniu pewnej
          świadomości otoczenia, a definiuje się ją także jako zdolność gry do wywoływania uczucia bycia
          częścią, czy "obecności" w wirtualnym środowisku gry. Sugeruje się, że większość regularnych graczy
          doświadcza pewnego stopnia imersji\cite{development_of_game}. Imersja (również nazywana "imersją
          sensoryczną i wyobrażeniową") jest oceniana za pomocą szeregu czynników odzwierciedlających siłę
          poczucia połączenia z grą\cite{validation_of_ge_scales}.
          Istnieją dyskusje na temat prawidłowej specyfikacji konstruktu imersji - czy powinien być modelowany
          jako refleksyjno-refleksyjny, czy refleksyjno-formatywny\cite{eng_in_games}.
    \item[Obecność] w kontekście gier wideo i wirtualnej rzeczywistości jest pojęciem, które
          wciąż ewoluuje i poszukuje ostatecznej definicji, jednak zazwyczaj jest określana w następujący
          sposób: 1. Bycie w normalnym stanie świadomości oraz 2. Doświadczanie poczucia znajdowania się
          wewnątrz wirtualnego środowiska\cite{development_of_game}. Większość, ale nie wszyscy gracze, prawdopodobnie mają zdolność do
          doświadczania obecności w odpowiednich warunkach\cite{development_of_game}. Poczucie obecności w
          alternatywnym środowisku może wynikać ze stymulacji sensorycznej, natomiast nie jest to pewne,
          dlatego w tym przypadku odwołać się można do koncepcji przepływu Csikszentmihalyi'ego (1997)\cite{measuring_narrative}.
          Ogólniej, obecność wiąże się z emocjonalnym zaangażowaniem w grę\cite{validation_of_ge_scales}.
    \item[Przepływ / trans] najczęściej występuje w literaturze anglojęzycznej pod pojęciem \textit{flow}.
          Przepływ można postrzegać jako głębokie, immersyjne doświadczenie, które wynika z zaangażowania się
          jednostki w zadanie o odpowiedniej równowadze pomiędzy wyzwaniem a poziomem umiejętności użytkownika\cite{measuring_engagement}.
          Przepływ i rozgrywka w grach są często łączone w kontekstach, gdzie użytkownik napotyka znaną formalną
          strukturę (gry), ale także nową zawartość stworzoną przez projektowanie systemu i wybory użytkownika
          w ramach tego systemu\cite{measuring_engagement}. Przepływ to termin opisujący uczucie przyjemności, które występuje, gdy osiąga
          się równowagę pomiędzy umiejętnościami a wyzwaniem w procesie wykonywania wewnętrznie nagradzającej
          aktywności\cite{development_of_game}. Posiadanie określonego celu i natychmiastowej informacji zwrotnej o wynikach zwiększa
          prawdopodobieństwo wystąpienia przepływu, a bycie w stanie przepływu wydaje się zwiększać uczenie
          się. Stany przepływu obejmują również uczucie kontroli, bycia jednością z aktywnością i doświadczanie
          zniekształceń czasu. Cechy te są dość podobne do niektórych anegdotycznych relacji zaangażowanych
          graczy wideo. Ponieważ wiąże się z doświadczaniem zmienionego stanu świadomości, doświadczenie
          przepływu może być nieco mniej powszechne niż imersja czy obecność\cite{development_of_game}. Z perspektywy modeli mentalnych
          przepływ lub przeniesienie do narracji występuje wtedy, gdy czytelnik lub widz całkowicie skupia się
          na czynności zrozumienia - tworzenia i aktualizowania modeli mentalnych reprezentujących historię\cite{measuring_narrative}.
    \item[Absorpcja] to termin używany do opisania całkowitego zaangażowania w bieżące doświadczenie. W
          przeciwieństwie do imersji i obecności, a podobnie jak przepływ, stan absorpcji psychologicznej
          indukuje zmieniony stan świadomości\cite{development_of_game}. W tym zmienionym stanie występuje rozdział myśli, uczuć i
          doświadczeń, a afekt jest mniej dostępny dla świadomości. Ogólną skłonność jednostki do wchodzenia
          w stan absorpcji psychologicznej można konceptualizować jako cechę, podczas gdy doświadczenie
          wchodzenia w stan absorpcji psychologicznej w określonej aktywności najlepiej postrzegać jako stan
          przejściowy\cite{development_of_game}. Niedawne badania powróciły do modelu zaangażowania w grę, w którym autor zrewidował
          oryginalny model, identyfikując, że absorpcja jest zasadniczo formą przepływu. Dlatego lepiej
          wykluczyć 'absorpcję' z modelu zaangażowania w grę\cite{eng_in_games}.
    \item[Dysocjacja] znana jest jako objaw kliniczny występujący u osób cierpiących na traumę, natomiast
          naturalnie występuję jej niepatologiczna forma\cite{development_of_game}. Najbardziej powszechnym przykładem dysocjacji
          niepatologicznej u dorosłych jest "hipnoza drogowa". Kierowcy wchodzą w stan absorpcji w
          niezwiązanej aktywności poznawczej. Równocześnie, konieczne obowiązki związane z manewrowaniem
          pojazdem są dalej wykonywane, pomimo że procesy mentalne związane z prowadzeniem są oddzielone
          od świadomej myśli\cite{development_of_game}.
\end{description}

\section{Sposoby pomiaru zaangażowania gracza}\label{section:ch3_2}

rioefeiofmeoi