\graphicspath{{chapters/chapter3/imgs/}}

\chapter{Zaangażowanie gracza}\label{chapter:ch3}

By móc określić wpływ rozwiązań sztucznej inteligencji na angażującą narrację wewnątrz gry należy
zdefiniować w jaki sposób określić i mierzyć można zaangażowanie gracza. W tej sekcji podjęta
zostanie próba skategoryzowania zaangażowania i wybrania najodpowiedniejszej metody do wykorzystania
w ramach eksperymentu.

\section{Definicje i rodzaje zaangażowania}\label{section:ch3_1}

Pomiar i ocena zaangażowania graczy w gry wideo jest niezwykle skomplikowanym zadaniem.
Podstawową przeszkodą jest to, że zaangażowanie nie jest zjawiskiem jednowymiarowym\cite{eng_in_games}, łatwym do
uchwycenia\cite{measuring_user_exp}. Stanowi raczej złożony, subiektywny stan umysłu, na który składać się
może wiele elementów. W literaturze naukowej poświęconej temu zagadnieniu można najczęściej spotkać się z terminami
takimi jak: imersja, obecność, przepływ / trans, absorpcja psychologiczna czy dysocjacja. Podstawowym problemem
jest fakt, że doświadczenia przeżywane podczas rozgrywki różnią się ze względu na gatunek gry, na stan
emocjonalny / psychiczny grającego czy też ze wględu na jego charakter\cite{measuring_user_exp}.

Kolejnym kluczowym utrudnieniem jest fakt, iż zaangażowanie oparte jest na
nieświadomych procesach poznawczych i emocjonalnych, do których trudno uzyskać
introspekcyjny dostęp\cite{measuring_user_exp}. Wymaganie od gracza, by w trakcie rozgrywki analizował i werbalizował swoje
przeżycia, nieuchronnie zakłóca i niszczy stan umysłu będący celem badań.
Zaangażowanie jest więc zjawiskiem dość ulotnym, które może w pełni przeminąć po wyciągnięciu grającego
ze sfery skupienia\cite{measuring_user_exp}.

Problemy pojawiają się również przy próbach retrospektywnego opisu i oceny zaangażowania po zakończeniu
sesji gry. Brakuje w tym obszarze nauki pewnego wspólnego słownika, dzięki któremu uczestnicy badań mogliby
w sposób jednoznaczny zwerbalizować subtelności tego złożonego doświadczenia\cite{measuring_user_exp}.
Ograniczeni jesteśmy do używania bardzo ogólnych terminów takich jak "frajda",
"zaangażowanie" czy "zaabsorbowanie", które nie oddają do końca bogactwa i głębi przeżywanych
stanów\cite{measuring_user_exp}.

Co więcej, zaangażowanie jest zjawiskiem silnie zależnym od otaczającego ją kontekstu, na który
składać się może sama rozgrywka ale i również interakcje społeczne czy fizyczne środowisko, w
którym znajduje się grający. Wszelkie próby jej wyizolowania i uproszczenia do czysto indywidualnego
doświadczenia gruntownie zniekształcają jego naturę.

W badaniach nad zaangażowaniem istotne są prace, które pomogły zdefiniować, gdzie ono rezyduje w
odniesieniu do szerokiego spektrum wymiarów lub stanów poznawczych i emocjonalnych, takich jak motywacja
i poczucie własnej skuteczności. Teoria samookreślenia pomaga wyjaśnić między innymi w jaki
sposób motywacja wewnętrzna - napędzana czynnikami takimi jak potrzeba kompetencji - jest ważnym
prekursorem zaangażowania. Zaangażowanie jest jednak czymś zgoła odmiennym od motywacji\cite{measuring_engagement}.
Można je postrzegać jako serię czasowych interakcji podczas wykonywania zadania, podczas gdy motywacja to
bardziej uniwersalne osobiste nastawienie wobec nauki/zadania. Połączenie tych dwóch stanów, czyli
zaangażowanie osoby w zadanie, do którego jest ona zmotywowana, może skutkować poczuciem satysfakcji
i chęci ponownego zaangażowania się w to zadanie\cite{measuring_engagement}.

Przedstawione zostaną najczęściej występujące w literaturze terminy pokrewne lub stanowiące część
zaangażowania:

\begin{itemize}
      \item Imersja (ang. \textit{Immersion})
      \item Obecność (ang. \textit{Presence})
      \item Przepływ / trans  (ang. \textit{Flow})
      \item Absorpcja (ang. \textit{Absorption})
      \item Dysocjacja (ang. \textit{Dissociation})
\end{itemize}

\begin{description}
      \item[Imersja] (zanurzenie) w grach wideo jest szeroko dyskutowanym pojęciem w pracach
            naukowych. Imersja zazwyczaj opisuje doświadczenie zaangażowania się w grę, przy zachowaniu pewnej
            świadomości otoczenia, a definiuje się ją także jako zdolność gry do wywoływania uczucia bycia
            częścią, czy "obecności" w wirtualnym środowisku gry. Przewiduje się, że niemal każdy przeciętny grający
            doświadcza pewnego stopnia imersji\cite{development_of_game}. Imersja (również nazywana "imersją
            sensoryczną i wyobrażeniową") oznaczać może również jak silne połączenie z grą odczuwał gracz\cite{validation_of_ge_scales}.
            Istnieją dyskusje na temat prawidłowej specyfikacji konstruktu imersji - czy powinien być modelowany
            jako refleksyjno-refleksyjny, czy refleksyjno-formatywny\cite{eng_in_games}.
      \item[Obecność] w kontekście gier wideo i wirtualnej rzeczywistości jest pojęciem, które
            wciąż ewoluuje i oczekuje na ostateczną definicję, jednak zazwyczaj jest określana jako połączenie
            tych dwóch aspektów: 1. Bycie w normalnym stanie świadomości oraz 2. Doświadczanie poczucia znajdowania się
            wewnątrz wirtualnego środowiska\cite{development_of_game}. Większość, ale nie wszyscy gracze, prawdopodobnie mają zdolność do
            doświadczania obecności w odpowiednich warunkach\cite{development_of_game}. Poczucie obecności w
            alternatywnym środowisku może wynikać ze stymulacji sensorycznej\cite{measuring_narrative}.
            Ogólniej, obecność wiąże się z emocjonalnym zaangażowaniem w grę\cite{validation_of_ge_scales}.
      \item[Przepływ / trans] najczęściej występuje w literaturze anglojęzycznej pod pojęciem \textit{flow}.
            Przepływ można postrzegać jako głębokie, imersyjne doświadczenie, które wynika z zaangażowania się
            osoby w zadanie o odpowiedniej równowadze pomiędzy wyzwaniem a poziomem umiejętności użytkownika\cite{measuring_engagement}.
            Przepływ i rozgrywka w grach są często łączone w kontekstach, gdzie użytkownik napotyka znaną (przykładowo
            z innych tytułów) strukturę rozgrywki natomiast doświadcza nowej treści (fabularnej,
            audio-wizualnej itd.)\cite{measuring_engagement}. Przepływ to termin opisujący uczucie przyjemności, które występuje, gdy osiąga
            się równowagę pomiędzy umiejętnościami a wyzwaniem w procesie wykonywania wewnętrznie nagradzającej
            aktywności\cite{development_of_game}. Posiadanie określonego celu i natychmiastowej informacji zwrotnej o wynikach zwiększa
            prawdopodobieństwo wystąpienia przepływu, a bycie w stanie przepływu wydaje się mieć wpływ na przebieg nauki
            użytkownika. Stany przepływu obejmują również uczucie kontroli, bycia jednością z aktywnością i doświadczanie
            zniekształceń czasu. Ponieważ wiąże się z doświadczaniem zmienionego stanu świadomości, doświadczenie
            przepływu może być nieco mniej powszechne niż imersja czy obecność\cite{development_of_game}. Z perspektywy modeli mentalnych
            przepływ do narracji występuje wtedy, gdy odbiorca całkowicie skupia się na czynności zrozumienia -
            tworzenia i aktualizowania modeli mentalnych reprezentujących historię\cite{measuring_narrative}.
      \item[Absorpcja] to termin używany do opisania całkowitego zaangażowania w bieżące doświadczenie. W
            przeciwieństwie do imersji i obecności, a podobnie do przepływu, stan absorpcji psychologicznej
            wiąże się ze zmienionym stanem świadomości\cite{development_of_game}. Ogólną skłonność osoby do wchodzenia
            w stan absorpcji psychologicznej można określać jako cechę, podczas gdy doświadczenie
            wchodzenia w ten stan w określonej aktywności najlepiej postrzegać jako stan przejściowy\cite{development_of_game}.
            Niektóre badania dotyczące modeli zaangażowania w grę twierdzą, że absorpcja jest zasadniczo formą
            przepływu. Wtedy też lepiej wykluczyć ją z ostatecznego modelu zaangażowania w grę\cite{eng_in_games}.
      \item[Dysocjacja] znana jest jako objaw kliniczny występujący u osób cierpiących na traumę, natomiast
            naturalnie występuję również jej niepatologiczna forma\cite{development_of_game}.
            Najbardziej powszechnym przykładem dysocjacji niepatologicznej u dorosłych jest "hipnoza drogowa".
            Kierowcy wchodą wtedy w stan absorpcji z aktywnością niezwiązaną z prowadzeniem.
            Równocześnie, konieczne obowiązki związane z manewrowaniem
            pojazdem są dalej wykonywane, pomimo że procesy mentalne związane z prowadzeniem są oddzielone
            od świadomej myśli\cite{development_of_game}.
\end{description}

\section{Sposoby pomiaru zaangażowania gracza}\label{section:ch3_2}

Istnieje wiele różnych podejść do pomiaru zaangażowania gracza, które różnią się zarówno zaletami,
jak i wadami. Metody fizjologiczne, takie jak monitorowanie tętna, częstości oddychania, aktywności
mięśniowej (elektromiografia), aktywności korowej (elektroencefalografia) oraz aktywności elektrycznej
skóry, zapewniają większą obiektywność, ale są zwykle droższe (pod względem czasowym i finansowym) oraz
trudniejsze w interpretacji\cite{validation_of_ge_scales}.
Analiza zachowań graczy w grze (telemetria) również oferuje obiektywność,
ale jako stosunkowo nowy obszar zainteresowania, wciąż stanowi wyzwanie redukcja złożoności poprzez
profilowanie oraz powiązanie zachowań w grze z subiektywnym doświadczeniem gracza. Z drugiej strony,
bardziej subiektywne metody oceny zaangażowania gracza, takie jak wywiady, grupy fokusowe, sondy w grze oraz
kwestionariusze, są stosunkowo tańszymi alternatywami i charakteryzują się mniejszymi trudnościami w
interpretacji niż pomiary fizjologiczne czy telemetria\cite{validation_of_ge_scales}.
Wywiady, grupy fokusowe i sondy w grze oferują stosunkowo głębokie wglądy, ale trudno je przeprowadzić na
dużą skalę. W przeciwieństwie do nich, kwestionariusze można łatwo dystrybuować wśród bardzo dużych grup,
a chociaż dostarczają mniej dogłębnych informacji niż inne subiektywne metody, umożliwiają skoncentrowanie
się na konkretnych aspektach zaangażowania gracza\cite{validation_of_ge_scales}.

W celu zewaluowania zaangażowania graczy w eksperymencie zdecydowano się na formę kwestionariusza.
Wybór konkretnego narzędzia ułatwiła praca Normana\cite{geq}, który porównał ze sobą dwa modele kwestionariusza
zaangażowania w grę (Game Engagement/Experience Questionnaire - GEQ)\cite{development_of_game}\cite{game_exp_quest}. Ostatecznie wybrano model
zaproponowany przez Brockmyera i współpracowników\cite{development_of_game}, który charakteryzuje się kilkoma istotnymi cechami.
Po pierwsze, podejście Brockmyera\cite{development_of_game} koncentruje się na ocenie skłonności pojedynczego dziecka do
zaangażowania się w grę wideo, a nie na ocenie samej gry\cite{geq}. Zamiast generować dużą liczbę pozycji i usuwać
zbędne lub niezrozumiałe dla uczestników na podstawie wyników empirycznych, Brockmyer i in.\cite{development_of_game} zaczęli od
zaledwie 10 pozycji z pięciostopniową skalą ocen. Dodatkowe pozycje zostały stworzone, aby lepiej
odzwierciedlić zaangażowanie, a następnie poddane ocenie na dwóch różnych próbach: 213 uczniów szkoły
średniej i 51 studentów. Analizy wykazały potrzebę dodania kolejnych pozycji, doprowadzając do powstania
19-punktowego kwestionariusza. Mimo że nie wyjaśniono, skąd pochodziły nowe pozycje, kwestionariusz ten
wykazuje dobre właściwości psychometryczne\cite{geq}, z współczynnikiem alfa Cronbacha 0,85 oraz estymacją
rzetelności osoby 0,83 i rzetelności pozycji 0,96 w modelu Rascha\cite{development_of_game}. Ponadto, Brockmyer i współpracownicy\cite{development_of_game}
skupili się na jednowymiarowym kontinuum określanym jako "zaangażowanie", co jest główną zaletą ich
pracy\cite{geq}. W celu zapewnienia walidacji zbieżnej, w pierwszym badaniu wykorzystano dopasowanie pomiędzy
teoretycznym oraz empirycznym uporządkowaniem trudności pozycji oraz zgodność pozycji i kategorii skali
ocen z jednowymiarowym pomiarem\cite{geq}. Autorzy przyznają jednak, że rekrutowali częstych graczy płci męskiej,
aby zwiększyć prawdopodobieństwo doświadczenia głębokiego zaangażowania w granie w sytuacji
laboratoryjnej, co może stanowić pewne ograniczenie\cite{geq}.

% TODO - dopisać, że niektóre pytania zostały zastąpione przez game_exp_quest
% bo lepiej oddawały nacisk na narrację + wkleić kwestio jako tabelkę i cacy rozdzialik