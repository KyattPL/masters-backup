\graphicspath{{chapters/chapter3/imgs/}}

\chapter{Zaangażowanie gracza}\label{chapter:ch3}

By móc określić wpływ rozwiązań sztucznej inteligencji na angażującą narrację wewnątrz gry należy
zdefiniować w jaki sposób określić i mierzyć można zaangażowanie gracza. W tej sekcji podjęta
zostanie próba skategoryzowania zaangażowania i wybrania najodpowiedniejszej metody do wykorzystania
w ramach eksperymentu.

\section{Definicje i rodzaje zaangażowania}\label{section:ch3_1}

Pomiar i ocena zaangażowania graczy w gry wideo jest niezwykle skomplikowanym zadaniem.
Podstawową przeszkodą jest to, że zaangażowanie nie jest zjawiskiem jednowymiarowym\cite{eng_in_games}, łatwym do
uchwycenia\cite{measuring_user_exp}. Stanowi raczej złożony, subiektywny stan umysłu, na który składają się różne komponenty
takie jak: immersja, obecność, przepływ (flow), absorpcja psychologiczna czy dyssocjacja. Brak jest
powszechnie akceptowanej definicji i sposobu operacjonalizacji tego konstruktu\cite{measuring_user_exp}.

Kolejnym kluczowym utrudnieniem jest fakt, iż doświadczenie zaangażowania oparte jest na
nieuświadomionych, podświadomych procesach poznawczych i emocjonalnych, do których trudno uzyskać
introspekcyjny dostęp\cite{measuring_user_exp}. Wymaganie od gracza, by w trakcie rozgrywki analizował i werbalizował swoje
przeżycia, nieuchronnie zakłóca i niszczy naturalny przebieg tego delikatnego stanu umysłu.
Zaangażowanie jest więc zjawiskiem "ulotnym" i efemerycznym, które rozpada się pod wpływem przymusowej
uważności\cite{measuring_user_exp}.

Problemy pojawiają się również przy próbach retrospektywnego opisu i oceny zaangażowania po zakończeniu
sesji gry. Ludziom brakuje wspólnego, wyspecjalizowanego słownika, który pozwoliłby precyzyjnie
zwerbalizować subtelności i niuanse tego złożonego doświadczenia\cite{measuring_user_exp}.
Ograniczeni jesteśmy do używania bardzo ogólnych, niedookreślonych terminów takich jak "frajda",
"zaangażowanie" czy "zaabsorbowanie", które nijak nie oddają bogactwa i głębi przeżywanych
stanów\cite{measuring_user_exp}.

Co więcej, zaangażowanie jest zjawiskiem silnie ukontekstowionym, zanurzonym w wielu wzajemnie
przenikających się warstwach - samej rozgrywki, interakcji społecznych oraz fizycznego środowiska, w
którym odbywa się gra. Wszelkie próby jego wyizolowania i uproszczenia do czysto indywidualnego
doświadczenia gruntownie zniekształcają i wypaczają jego naturę.

W badaniach nad zaangażowaniem istotne są prace, które pomogły zdefiniować, gdzie ono rezyduje w
odniesieniu do szerokiego spektrum wymiarów lub stanów poznawczych i afektywnych, takich jak motywacja
i poczucie własnej skuteczności. Wskazują one, że teoria samookreślenia pomaga wyjaśnić, w jaki
sposób motywacja wewnętrzna - napędzana czynnikami takimi jak potrzeba kompetencji - jest ważnym
prekursorem zaangażowania. Zaangażowanie jest jednak konstruktem odrębnym od motywacji\cite{measuring_engagement}. Można je
postrzegać jako serię czasowych interakcji podczas wykonywania zadania, podczas gdy motywacja to
bardziej globalna osobista orientacja wobec nauki/zadania. Interakcja między nimi może utworzyć pętlę
sprzężenia zwrotnego, w której doświadczenie z zadaniem kształtuje bardziej zbliżone do stanu elementy
poczucia własnej skuteczności i motywacji, które z kolei wpływają na chęć ponownego zaangażowania się
użytkownika w zadanie\cite{measuring_engagement}.

Przedstawione zostaną najczęściej występujące w literaturze terminy pokrewne lub stanowiące część
zaangażowania:

\begin{itemize}
    \item Immersion
    \item Presence
    \item Flow
    \item Absorption
    \item Dissociation
\end{itemize}

\subsubsection*{Immersion}

\lipsum[1]

\begin{description}
    \item[Immersion] A może w ten sposób lepiej by to wyglądało? Jest szansa niewątpliwie. A no ale
          jeszcze brakuje trochę tekstu dodatkowego żeby stwierdzić jak to wygląda przy kilku wierszach.
\end{description}

\section{Sposoby pomiaru zaangażowania gracza}\label{section:ch3_2}

rioefeiofmeoi