\graphicspath{{chapters/chapter4/imgs/}}

\chapter{Planowany eksperyment}\label{chapter:ch4}

W tej sekcji przedstawiony zostanie pełny plan eksperymentu, w tym: projekt gry przeglądarkowej
z gatunku \textit{visual novel}, opis implementacji i wykorzystania generatywnych agentów
opartych na dużych modelach językowych oraz sam przebieg badań.

\section{Projekt gry wykorzystanej w eksperymencie}\label{section:ch4_1}

Celem gry ma być oczywiście zbadanie wpływu wykorzystania generatywnych agentów na zaangażowanie
grającego w narrację. W związku z tym rozgrywka powinna kłaść nacisk przede wszystkim na 
przedstawianie treści fabularnych, pozostawiając walory estetyczne czy mechaniki gry na drugim
planie. Dodatkowo, całe doświadczenie powinno mieścić się w przedziale 15-30min i być dostępne
w łatwy sposób (bez instalacji). Z uwagi na te założenia zdecydowano się na utworzenie gry
dwuwymiarowej z gatunku \textit{visual novel} (Patrz sekcja poniżej).

\subsection*{Gatunek \textit{visual novel}}\label{subsection:ch4_1_1}

Forma \textit{visual novel} (z ang. \textit{powieść wizualna}) jest najczęściej uznawana jako
gatunek gier komputerowych, choć niektórzy dostrzegają w niej zupełnie odrębne od gier 
medium\cite{tvtropes_visual_novel}. 

Kluczowymi cechami \textit{visual novel} są prezentacja tekstu za pomocą okienek dialogowych, które gracz 
musi klikać, aby przejść dalej, oraz statyczne grafiki przedstawiające postacie i otoczenie. Choć 
powieści wizualne często zawierają elementy multimedialne, takie jak animacje, muzyka czy dubbing, 
to nie są one ich kluczowymi składnikami\cite{tvtropes_visual_novel}.

\textit{Visual novel} koncentrują się przede wszystkim na prezentacji narracji, z niewielką lub zerową 
ilością rozgrywki. Wiele z nich oferuje nieliniową, rozgałęziającą się fabułę z wieloma 
zakończeniami i systemem wyborów wpływających na dalszy przebieg wydarzeń\cite{tvtropes_visual_novel}. 
Z drugiej strony, istnieją też powieści wizualne
pozbawione jakiejkolwiek rozgrywki i rozgałęzień fabularnych, określane mianem 
\textit{kinetic novel}\cite{tvtropes_kinetic_novel}.

Ogólnie powieści wizualne wyróżniają się dominacją narracji przedstawianej za pomocą tekstu i grafik nad 
rozgrywką. Kryterium odróżniające je od gier przygodowych jest stopień, w jakim faktycznie 
wykorzystują mechaniki gry i gameplay w stosunku do narracji\cite{tvtropes_visual_novel}.

\subsection*{Projekt gry}

Gra została opracowana przy wykorzystaniu silnika Ren'Py w wersji 8.2.1, opartego na języku Python. 
Ren'Py to popularne narzędzie służące do tworzenia gier tego rodzaju, oferujące zaawansowane 
możliwości pisania scenariuszy, zarządzania obrazami i dźwiękiem oraz tworzenia systemów wyborów 
i rozgałęzień fabularnych.

Wszystkie niezbędne zasoby graficzne pozyskane zostały ze społeczności twórców niezależnych na 
platformie itch.io. Grafiki postaci i tła zostały wyprodukowane przez twórców 
LinXueLian (https://linxuelian.itch.io/) oraz Potat0Master (https://potat0master.itch.io/), 
specjalizujących się w tego typu materiałach na potrzeby \textit{visual novel} i gier przygodowych.

Po zakończeniu prac, gotowy produkt został wyeksportowany do postaci pliku wykonywalnego 
kompatybilnego z przeglądarkami internetowymi. Takie rozwiązanie umożliwia graczom pobieranie i 
uruchamianie tytułu bezpośrednio ze stron www, bez konieczności instalowania dodatkowego 
oprogramowania.

Finalną wersję gry opublikowano na platformie dystrybucji cyfrowej itch.io, która stała się 
głównym kanałem udostępniania produkcji odbiorcom. Itch.io jest popularnym hubem dla niezależnych 
twórców gier, w tym deweloperów \textit{visual novel} wykorzystujących silniki takie jak Ren'Py.

\subsection*{Zarys fabularny}

lorfloefleofe

\section{Opis generatywnych agentów}\label{section:ch4_2}

faefaef

\subsection*{Struktury wykorzystywane przez inworld.ai}

lorfloefleofe

\subsection*{Opis wykreowanych postaci}

lorfloefleofe

\subsection*{Wykorzystanie agentów w praktyce}

lorfloefleofe

\section{Zaplanowany przebieg eksperymentu}\label{section:ch4_3}

pofkepfokawpefkwpeofk

\subsection*{Forma eksperymentu}

lorfloefleofe

\subsection*{Gromadzone dane}

lorfloefleofe