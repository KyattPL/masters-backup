% !TEX encoding = UTF-8 Unicode 
% !TEX root = praca.tex
\chapter{Narracja w grach}\label{chapter:ch1}

Niniejszy rozdział ma na celu dokonanie przeglądu gier komputerowych na przestrzeni lat, ze
szczególnym naciskiem na ewolucję sposobów oraz form narracji przedstawianych w tych grach.
Wyszczególnione zostaną również naistotniejsze struktury i rodzaje narracji, które są współcześnie
wykorzystywane. Dodatkowo, nastąpi krótki przegląd najpopularniejszych technik prezentacji narracji.
Na koniec nakreślone zostaną systemy dialogowe wykorzystywane przez gry komputerowe.

\section{Historia narracji w grach komputerowych}\label{section:ch1_1}

Aby zrozumieć istotę narracji w grach komputerowych, należy przede wszystkim określić
co może kryć się pod tym pojęciem. Pozwoli to dokonać przeglądu wybranych tytułów
i wyciągnąć z tego przeglądu wnioski. Żeby udowodnić rozwój w sposobie prezentowania narracji
na przestrzeni lat, prześledzone zostały części jednej z serii gier --- \textit{"Final Fantasy"} ---
wydawanej od roku 1987.

\subsection{Definicja narracji}\label{subsection:ch1_1_1}

Pojęcie narracji i samo jej występowanie w grach komputerowych jest kwestią sporną
w literaturze od lat. Barry Ip, w swojej pracy \cite{narrative_structures}, dokonuje wyróżnienia trzech słów ściśle
powiązanych ze sobą: \textit{historia}, \textit{fabuła} oraz \textit{narracja}. Na potrzeby jego
badań historia zdefiniowana została następująco:

\begin{quotation}
	\ldots \textit{sekwencja zdarzeń obejmujących byty.} \cite{narrative_structures}
\end{quotation}

Związana z historią jest również fabuła, która została określona przez Arystotelesa jako:

\begin{quotation}
	\ldots \textit{organizacja zdarzeń.} \cite{narrative_structures}
\end{quotation}

Sama narracja, ściśle powiązana z dwoma poprzednimi terminami, wyrażona została w sposób
następujący:

\begin{quotation}
	\ldots \textit{reprezentacja zdarzenia lub serii zdarzeń.} \cite{narrative_structures}
\end{quotation}

W ramach tejże pracy, można przyjąć wszystkie te pojęcia jako istotne i na tyle bliskie
siebie, że mogą się zastępować.

Jakub Majewski sugeruje, że debatowanie nad istnieniem narracji jest odpowiednie dla niektórych
gier, a dla niektórych nie \cite{theorising_narrative}. Rozdzielenie bowiem tych form
przekazu, które można zaliczyć do treści fabularnej, nie jest takie oczywiste. Przytoczyć można
przykład \textit{Space Invaders} (1977) --- gra nie przytacza żadnego opisu w formie tekstowej,
skupiając się wyłącznie na rozgrywce. Na podstawie samego tytułu można jednak
przypuścić, że dokonuje się pewnego rodzaju inwazja, a stoją za nią przybysze z kosmosu \cite{theorising_narrative}.

TODO \ldots

Do budowania narracji w grach wykorzystane mogą być wzorce znane z literatury. Przykładem takiego wzorca jest
\textit{"Podróż bohatera"}\cite{narrative_structures}, który opisuje 12 kluczowych etapów, odgrywających
istotną rolę w budowie angażujących historii (Tabela \ref{tab1:ch1_1_1}).

\begin{table}[h!]
	\caption{Dwanaście etapów...}
	\label{tab1:ch1_1_1}
	\begin{center}
		\begin{tabular}{p{1.5in} p{4in}}
			\hline
			Stage                                 & Description                                                         \\
			\hline
			1. The ordinary world                 & The player first meets the hero and is introduced to the hero’s
			background, typically via the back story                                                                    \\
			2. The call to adventure              & A hint that the hero will be leaving the ordinary world to begin a
			new adventure. This stage acts as a catalyst that triggers off the
			main storyline                                                                                              \\
			3. The refusal of the call            & In the traditional structure of the monomyth, the hero will turn
			down the initial offer to leave the ordinary world to begin a
			quest, usually as a moment of doubt or uncertainty                                                          \\
			4. The meeting with the mentor        & When the hero decides to take on the quest, the mentor
			provides her with the information needed to choose what action
			to take. Mentors can be anything which provides information—a
			bearded old man, hub, robot, library, past experiences, and so
			on                                                                                                          \\
			5. Crossing the first threshold       & The hero crosses from the safety of the ordinary world to a
			new, dangerous, and unknown world of the quest                                                              \\
			6. Tests, allies and enemies          & This phase is usually the largest part of the game story, as the
			player is introduced to all the major characters                                                            \\
			7. The approach to the innermost cave & This is where the hero finds the reward she seeks—such as
			gaining the essential skill, weapon, or mastery of everything she
			has come across up to this point. Typically, this is situated
			toward the end of the game. The main objective of this part of
			the story is to prepare the hero for the final battle                                                       \\
			8. The ordeal                         & This is where the hero faces the final battle with her nemesis or
			‘‘final boss.’’ The nemesis can appear as either a physical (person
			or object) or nonphysical (time, intensity, or difficulty) entity                                           \\
			9. The reward                         & Many games end at this point, when the enemy is defeated and
			the reward is usually an ending cut scene detailing what happens
			to the hero after her triumph                                                                               \\
			10. The road back                     & Some games will allow the player to return to the ordinary
			world after the reward but it may not be possible for the hero to
			integrate successfully into the old world                                                                   \\
			11. The resurrection                  & This part of the story addresses any unanswered questions, such
			as the consequences from the quest, potential conflicts that may
			arise for future sequels, or any tests the hero must face before
			the end. It can also be in the form of a final plot twist, as
			something unexpected by the audience                                                                        \\
			12. The return with the reward        & This is the last stage of the story, where the hero finally returns
			to the ordinary world and sees the benefits of her reward. The
			hero can compare her life before and after the quest to see how
			things have changed                                                                                         \\
			\hline
		\end{tabular}
	\end{center}
\end{table}

\newpage

Blisko powiązana z \textit{"Podróżą bohatera"} jest znana struktura trzech aktów opisana przez
Arystotelesa, która zakłada podział utworu na początek, środek i koniec\cite{narrative_structures}.

TODO ZAKOŃCZENIE TEJ PODSEKCJI \ldots

\lipsum[1]

\subsection{Przedstawienie narracji w grach na przestrzeni lat}\label{subsection:ch1_1_2}

gegege

\subsection{Prześledzenie rozwoju narracji na przykładzie serii Final Fantasy}\label{subsection:ch1_1_3}

goiemgoe

\section{Rodzaje narracji w grach komputerowych}\label{section:ch1_2}

efe

\subsection{Struktury narracji}\label{subsection:ch1_2_1}

fefef

\subsection{Rodzaje narracji}\label{subsection:ch1_2_2}

fefef

\subsection{Techniki przedstawienia narracji}\label{subsection:ch1_2_3}

ioegemo

\section{Systemy dialogowe w grach komputerowych}\label{section:ch1_3}

fefef

\subsection{Popularne systemy dialogowe}\label{subsection:ch1_3_1}

fefef

\subsection{Interaktywna fikcja - system poleceń}\label{subsection:ch1_3_2}

ortatraoitrmoi