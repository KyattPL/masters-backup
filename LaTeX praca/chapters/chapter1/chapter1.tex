% !TEX encoding = UTF-8 Unicode 
% !TEX root = praca.tex
\chapter{Narracja w grach}\label{chapter:ch1}

Niniejszy rozdział ma na celu dokonanie przeglądu gier komputerowych na przestrzeni lat, ze
szczególnym naciskiem na ewolucję sposobów oraz form narracji przedstawianych w tych grach.
Wyszczególnione zostaną również naistotniejsze struktury i rodzaje narracji, które są współcześnie
wykorzystywane. Dodatkowo, nastąpi krótki przegląd najpopularniejszych technik prezentacji narracji.
Na koniec nakreślone zostaną systemy dialogowe wykorzystywane przez gry komputerowe.

\section{Historia narracji w grach komputerowych}\label{section:ch1_1}

Aby zrozumieć istotę narracji w grach komputerowych, należy przede wszystkim określić
co może kryć się pod tym pojęciem. Pozwoli to dokonać przeglądu wybranych tytułów
i wyciągnąć z tego przeglądu wnioski. Żeby udowodnić rozwój w sposobie prezentowania narracji
na przestrzeni lat, prześledzone zostały części jednej z serii gier --- \textit{"Final Fantasy"} ---
wydawanej od roku 1987.

\subsection{Definicja narracji}\label{subsection:ch1_1_1}

Pojęcie narracji i samo jej występowanie w grach komputerowych jest kwestią sporną
w literaturze od lat. Barry Ip, w swojej pracy \cite{narrative_structures}, dokonuje wyróżnienia trzech słów ściśle
powiązanych ze sobą: \textit{historia}, \textit{fabuła} oraz \textit{narracja}. Na potrzeby jego
badań historia zdefiniowana została następująco:

\begin{quotation}
	\ldots \textit{sekwencja zdarzeń obejmujących byty.} \cite{narrative_structures}
\end{quotation}

Związana z historią jest również fabuła, która została określona przez Arystotelesa jako:

\begin{quotation}
	\ldots \textit{organizacja zdarzeń.} \cite{narrative_structures}
\end{quotation}

Sama narracja, ściśle powiązana z dwoma poprzednimi terminami, wyrażona została w sposób
następujący:

\begin{quotation}
	\ldots \textit{reprezentacja zdarzenia lub serii zdarzeń.} \cite{narrative_structures}
\end{quotation}

W ramach tejże pracy, można przyjąć wszystkie te pojęcia jako istotne i na tyle bliskie
sobie, że mogą się zastępować.

Jakub Majewski sugeruje, że debatowanie nad istnieniem narracji jest odpowiednie dla niektórych
gier, a dla niektórych nie \cite{theorising_narrative}. Rozdzielenie bowiem tych form
przekazu, które można zaliczyć do treści fabularnej, nie jest takie oczywiste. Przytoczyć można
przykład \textit{Space Invaders} (1977) --- gra nie przytacza żadnego opisu w formie tekstowej,
skupiając się wyłącznie na rozgrywce. Na podstawie samego tytułu można jednak
przypuścić, że dokonuje się pewnego rodzaju inwazja, a stoją za nią przybysze z kosmosu \cite{theorising_narrative}.

TODO \ldots

Do budowania narracji mogą służyć też techniki znane z literatury. Przykładem takiej techniki jest
\textit{"Podróż bohatera"}\cite{narrative_structures}, która opisuje 12 kluczowych etapów, odgrywających
istotną rolę w budowie angażujących historii (Tabela 1).

\begin{table}[ht]
	\caption{Dwanaście etapów...}
	\label{tab1:ch1_1_1}
	\begin{center}
		\begin{tabular}{l l}
			\hline
			Stage                 & Description \\
			\hline
			1. The ordinary world & bla bla     \\
			1. The ordinary world & bla bla     \\
			1. The ordinary world & bla bla     \\
			1. The ordinary world & bla bla     \\
			1. The ordinary world & bla bla     \\
			\hline
		\end{tabular}
	\end{center}
\end{table}

Blisko powiązana z \textit{"Podróżą bohatera"} jest znana struktura trzech aktów opisana przez
Arystotelesa, która zakłada podział utworu na początek, środek i koniec\cite{narrative_structures}.

TODO ZAKOŃCZENIE TEJ PODSEKCJI \ldots

\subsection{Przedstawienie narracji w grach na przestrzeni lat}\label{subsection:ch1_1_2}

gegege

\subsection{Prześledzenie rozwoju narracji na przykładzie serii Final Fantasy}\label{subsection:ch1_1_3}

goiemgoe

\section{Rodzaje narracji w grach komputerowych}\label{section:ch1_2}

efe

\subsection{Struktury narracji}\label{subsection:ch1_2_1}

fefef

\subsection{Rodzaje narracji}\label{subsection:ch1_2_2}

fefef

\subsection{Techniki przedstawienia narracji}\label{subsection:ch1_2_3}

ioegemo

\section{Systemy dialogowe w grach komputerowych}\label{section:ch1_3}

fefef

\subsection{Popularne systemy dialogowe}\label{subsection:ch1_3_1}

fefef

\subsection{Interaktywna fikcja - system poleceń}\label{subsection:ch1_3_2}

ortatraoitrmoi