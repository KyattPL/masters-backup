\chapter{Sposoby generowania narracji}\label{chapter:ch2}

Bla bla bla\dots

\section{Wykorzystanie algorytmów sztucznej inteligencji do kreowania narracji}\label{section:ch2_1}

Wyszczególnione zostaną najpopularniejsze zaawansowane algorytmy sztucznej inteligencji.

\subsubsection*{POP (Partially-ordered planning)}

Bla bla bla...

\subsubsection*{PEM (Player experience modelling)}

Bla bla bla...

\subsubsection*{NLP (Natural language processing)}

Bla bla bla...

\subsubsection*{NPC (Non-playable character)}

Bla bla bla...

\subsubsection*{MDP (Markov decision process)}

Bla bla bla...

Żeby nie być gołosłownym to przedstawione zostaną teraz trzy przykłady gier wykorzystujących
algorytmy sztucznej inteligencji

\subsubsection*{Facade}

Bla bla bla...

\subsubsection*{QuestVille}

Bla bla bla...

\section{Wykorzystanie dużych modeli językowych (LLM) do kreowania narracji}\label{section:ch2_2}

Bla bla bla duże modele językowe są coraz popularniesze i coraz lepsze, więc zaczynają być
wykorzystywane w coraz więcej miejscach. W tej sekcji przedstawiona zostanie krótka charakterystyka
dużych modeli językowych, potencjalne formy ich wykorzystania oraz przykład praktyczny.

\subsubsection*{Definicja i charakterystyka dużych modeli językowych}

Bla bla bla...

\subsubsection*{Potencjalne struktury wykorzystania dużych modeli językowych}

Bla bla bla...

\subsubsection*{Przykład zastosowania dużych modeli językowych}

Bla bla bla...