\chapter{Sposoby generowania narracji}\label{chapter:ch2}

W poprzednich sekcjach omówiona została historia oraz wykorzystywane w grach komputerowych rodzaje
narracji. W każdym z tych elementów istnieje jeden element wspólny: to ludzie odpowiadają za tworzenie
narracji, odpowiednie jej planowanie i prezentowanie odbiorcy. Związane z tym są oczywiście olbrzymie
koszty oraz duży nakład czasu poświęcony przez pracowników. Biorąc pod uwagę terminy stale goniące
producentów gier, nic dziwnego, że pewne zaplanowane fragmenty fabularne nie znajdują miejsca w końcowym
produkcie. Z tego względu, patrząc na postępujący rozwój technologiczny, nasuwa się pytanie ---
czy da się zarządzać narracją w grze komputerowej w sposób automatyczny? W ramach tej sekcji
przedstawione zostaną znane rozwiązania z dziedziny sztucznej inteligencji pomagające w kreowaniu fabuły
oraz nakreślony zostanie potencjał w tej sferze dużych modeli językowych.

\section{Wykorzystanie algorytmów sztucznej inteligencji do kreowania narracji}\label{section:ch2_1}

Mówiąc o ogólnym wykorzystaniu algorytmów sztucznej inteligencji można cofnąć się do bardzo prostych
rozwiązań wykorzystanych np. w "Pong" (Patrz sekcja \ref{subsection:ch1_1_2}), do technik generowania
proceduralnego (zwłaszcza poziomów) czy też do systemów rankingowych (np. system TrueSkill). Jako, że
nie są to metody stricte powiązane z narracją to nie zostaną one opisane bardziej szczegółowo.
Przedstawione natomiast będą kluczowe obszary wykorzystywane w grach: częściowo-uporządkowane planowanie
(ang. \textit{POP} - partially-ordered planning), modelowanie doświadczeń gracza (ang. \textit{PEM} -
player experience modelling), przetwarzanie języka naturalnego (ang. \textit{NLP} - natural language
processing), postać niegrywalna (ang. \textit{NPC} - non-playable character), proces decyzyjny Markowa
(ang. \textit{MDP} - Markov decision process).

\subsubsection*{POP (Partially-ordered planning)}

Bla bla bla...

\subsubsection*{PEM (Player experience modelling)}

Bla bla bla...

\subsubsection*{NLP (Natural language processing)}

Bla bla bla...

\subsubsection*{NPC (Non-playable character)}

Bla bla bla...

\subsubsection*{MDP (Markov decision process)}

Bla bla bla...

Żeby nie być gołosłownym to przedstawione zostaną teraz trzy przykłady gier wykorzystujących
algorytmy sztucznej inteligencji

\subsubsection*{Facade}

Bla bla bla...

\subsubsection*{QuestVille}

Bla bla bla...

\section{Wykorzystanie dużych modeli językowych (LLM) do kreowania narracji}\label{section:ch2_2}

Bla bla bla duże modele językowe są coraz popularniesze i coraz lepsze, więc zaczynają być
wykorzystywane w coraz więcej miejscach. W tej sekcji przedstawiona zostanie krótka charakterystyka
dużych modeli językowych, potencjalne formy ich wykorzystania oraz przykład praktyczny.

\subsubsection*{Definicja i charakterystyka dużych modeli językowych}

Bla bla bla...

\subsubsection*{Potencjalne struktury wykorzystania dużych modeli językowych}

Bla bla bla...

\subsubsection*{Przykład zastosowania dużych modeli językowych}

Bla bla bla...