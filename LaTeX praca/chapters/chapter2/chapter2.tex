\graphicspath{{chapters/chapter2/imgs/}}

\chapter{Sposoby generowania narracji}\label{chapter:ch2}

W poprzednich sekcjach omówiona została historia oraz wykorzystywane w grach komputerowych rodzaje
narracji. W każdym z tych elementów istnieje jeden element wspólny: to ludzie odpowiadają za tworzenie
narracji, odpowiednie jej planowanie i prezentowanie odbiorcy. Związane z tym są oczywiście olbrzymie
koszty oraz duży nakład czasu poświęcony przez pracowników. Biorąc pod uwagę terminy stale goniące
producentów gier, nic dziwnego, że pewne zaplanowane fragmenty fabularne nie znajdują miejsca w końcowym
produkcie. Z tego względu, patrząc na postępujący rozwój technologiczny, nasuwa się pytanie ---
czy da się zarządzać narracją w grze komputerowej w sposób automatyczny? W ramach tej sekcji
przedstawione zostaną znane rozwiązania z dziedziny sztucznej inteligencji pomagające w kreowaniu fabuły
oraz nakreślony zostanie potencjał w tej sferze dużych modeli językowych.

\section{Wykorzystanie algorytmów sztucznej inteligencji do kreowania narracji}\label{section:ch2_1}

Mówiąc o ogólnym wykorzystaniu algorytmów sztucznej inteligencji można cofnąć się do bardzo prostych
rozwiązań wykorzystanych np. w "Pong" (Patrz sekcja \ref{subsection:ch1_1_2}), do technik generowania
proceduralnego (zwłaszcza poziomów) czy też do systemów rankingowych (np. system TrueSkill). Jako, że
nie są to metody stricte powiązane z narracją to nie zostaną one opisane bardziej szczegółowo.
Przedstawione natomiast będą kluczowe obszary wykorzystywane w grach: częściowo-uporządkowane planowanie
(ang. \textit{POP} - partially-ordered planning), modelowanie doświadczeń gracza (ang. \textit{PEM} -
player experience modelling), przetwarzanie języka naturalnego (ang. \textit{NLP} - natural language
processing), postać niegrywalna (ang. \textit{NPC} - non-playable character), proces decyzyjny Markowa
(ang. \textit{MDP} - Markov decision process).

\subsubsection*{POP (Partially-ordered planning)}

Planowanie częściowo-uporządkowane jest skierowanym grafym acyklicznym, gdzie węzły są operacjami
(inaczej nazywanymi akcjami), które po wywołaniu zmieniają stan świata. Krawędzie przedstawiają
relacje przyczynowe i czasowe pomiędzy akcjami. Powiązanie przyczynowe $a_{i} \rightarrow^{c}a_{j}$
oznacza, że wykonanie akcji $a_{i}$ zmieni stan warunku $c$ na prawdziwy w świecie fabuły, a co za
tym idzie akcja $a_{j}$ zależna od tego warunku będzie możliwa do wykonania. Powiązanie czasowe
przedstawia ograniczenie porządkowe pomiędzy operacjami, gdzie jedna operacja musi być wykonana przed
inną\cite{game_ai_storytelling}. Przykładowa struktura fabularna zrealizowana za pomocą planowania
częściowo-uporządkowanego została przedstawiona na rysunku \ref{fig:ch2_1_pop}.

\begin{figure}[h]
    \centering
    \includegraphics[width=0.5\textwidth]{ch2_1_pop.png}
    \caption{Fabuła "Czerwonego Kapturka" zapisana za pomocą POP}
    \label{fig:ch2_1_pop}
\end{figure}

Za pomocą tej techniki kreować można rozbudowane plany fabularne, które mogą ulegać zmianie na
podstawie akcji podejmowanych przez gracza czy zmian zachodzących w świecie gry. Odpowiednie algorytmy
przeszukiwania nazywane \textit{"plannerami"} rozwiązują problem planowania, tzn. mając dany stan
świata, pewne atomowe akcje możliwe do wykonania przez grającego oraz założony cel, znajdują
odpowiednią sekwencję operacji, które doprowadzą do osięgnięcia tegoż celu\cite{game_ai_storytelling}.

Podstawowym problemem pojawiającym się przy wykorzystaniu metody POP jest to, że zarówno gracz jak i
potencjalnie inne niegrywalne postacie, mogą być zdolne do wywołania akcji w świecie gry, która
zagraża dalszemu przebiegowi fabularnemu zgodnego z planem\cite{characters_and_directors}. Wtedy
stosowane są odpowiednie techniki naprawcze, które prowadzą do mniej lub bardziej doskonałych rozwiązań.

\subsubsection*{PEM (Player experience modelling)}

Modelowanie doświadczeń gracza polega na zbieraniu danych behawioralnych czy też wydajnościowych
(punkty, czas, decyzje) ze względu na rozgrywkę za pomocą wielu modalności: mowy gracza, obrazów
(śledzenie ruchów ciała, mimiki twarzy, gałek ocznych) czy też sygnałów fizjologicznych (puls czy
przewodność skóry). Pomiar sygnałów fizjologicznych jest oczywiście problematyczny ze względu na
wykorzystanie dodatkowego sprzętu a zarazem inwazyjność przeszkadzającą w swobodnej rozgrywce
\cite{reusable_game_ai}.

W ramach tego obszaru sztuczna inteligencja objawia się zazwyczaj pod postacią sieci neuronowych czy
też drzew decyzyjnych, które pozwalają dokonywać klasyfikacji w zakresie\cite{reusable_game_ai}:

\begin{itemize}
    \item rozpoznawania emocji grającego - w ramach anagażujących systemów dialogowych
    \item balansowania rozgrywki - tak by gracz nie odczuwał frustracji ze względu na zbyt wysoki poziom
          trudności a zarazem by nie doznawał nudy ze względu na zbyt prostą rozgrywkę
    \item oceniania umiejętności gracza - do prowadzenia badań w sposób ukryty
\end{itemize}

\subsubsection*{NLP (Natural language processing)}

Bla bla bla...

\subsubsection*{NPC (Non-playable character)}

Bla bla bla...

\subsubsection*{MDP (Markov decision process)}

Bla bla bla...

Żeby nie być gołosłownym to przedstawione zostaną teraz dwa przykłady gier wykorzystujących
algorytmy sztucznej inteligencji

\subsubsection*{Facade}

Bla bla bla...

\subsubsection*{QuestVille}

Bla bla bla...

\section{Wykorzystanie dużych modeli językowych (LLM) do kreowania narracji}\label{section:ch2_2}

Bla bla bla duże modele językowe są coraz popularniesze i coraz lepsze, więc zaczynają być
wykorzystywane w coraz więcej miejscach. W tej sekcji przedstawiona zostanie krótka charakterystyka
dużych modeli językowych, potencjalne formy ich wykorzystania oraz przykład praktyczny.

\subsubsection*{Definicja i charakterystyka dużych modeli językowych}

Bla bla bla...

\subsubsection*{Potencjalne struktury wykorzystania dużych modeli językowych}

Bla bla bla...

\subsubsection*{Przykład zastosowania dużych modeli językowych}

Bla bla bla...