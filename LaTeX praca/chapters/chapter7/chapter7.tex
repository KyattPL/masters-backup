\graphicspath{{chapters/chapter7/imgs/}}

\chapter{Wyniki}\label{chapter:ch7}

W tym rozdziale przedstawione zostaną ostateczne wyniki przeprowadzonego eksperymentu, w tym: przedstawienie
demografii uczestników badania, dokonanie analizy ilościowej i jakościowej danych.

\section{Demografia}\label{section:ch7_1}

W badaniu wzięły udział 34 osoby, gdzie 14 uczestników wypełniło formularz A, a 20 osób formularz B.
Ogólny rozkład płci uczestników był stosunkowo wyrównany - w przypadku formularza A, 9 osób to kobiety, a 5 to
mężczyźni. Z kolei wśród wypełniających formularz B, 7 osób to kobiety, 12 to mężczyźni, a 1 osoba wybrała
opcję "Inna" w pytaniu o płeć. Te informacje można odczytać z tabeli \ref{tab1:ch7_1} i rysunku \ref{fig:ch7_demo1}.

\begin{table}[h!]
    \begin{center}
        \begin{tabular}{|l|r|r|}
            \hline
            Płeć      & Liczba osób & Procent całości \\
            \hline
            Kobieta   & 16          & 47,06\%         \\
            Mężczyzna & 17          & 50,00\%         \\
            Inne      & 1           & 2,94\%          \\
            \hline
        \end{tabular}
    \end{center}
    \caption{Uczestnicy badania ze względu na płeć}\label{tab1:ch7_1}
\end{table}

\begin{figure}[h!]
    \centering
    \includegraphics[width=0.9\textwidth]{demo1.png}
    \caption{Płeć uczestników badania z podziałem na formularz A i B}
    \label{fig:ch7_demo1}
\end{figure}

\newpage

Większość uczestników należała do grupy wiekowej 20-29 lat, co odzwierciedlają dane przedstawione w
tabeli \ref{tab1:ch7_2} i na rysunku \ref{fig:ch7_demo2} . Wśród wypełniających formularz A było 9 osób
w wieku 20-29 lat, a w przypadku formularza B - 11 osób. Uczestnicy reprezentowali jednak różne grupy
wiekowe, od nastolatków po osoby po 30. i 40. roku życia.

\begin{table}[h!]
    \begin{center}
        \begin{tabular}{|l|r|r|}
            \hline
            Przedział wiekowy & Liczba osób & Procent całości \\
            \hline
            15-19             & 4           & 11,76\%         \\
            20-24             & 12          & 35,29\%         \\
            25-29             & 8           & 23,53\%         \\
            30-34             & 6           & 17,65\%         \\
            35-39             & 3           & 8,82\%          \\
            40-45             & 1           & 2,94\%          \\
            \hline
        \end{tabular}
    \end{center}
    \caption{Uczestnicy badania ze względu na wiek}\label{tab1:ch7_2}
\end{table}

\begin{figure}[h!]
    \centering
    \includegraphics[width=0.9\textwidth]{demo2.png}
    \caption{Wiek uczestników badania z podziałem na formularz A i B}
    \label{fig:ch7_demo2}
\end{figure}

\newpage

Uczestnicy badania pochodzili z różnych kontynentów, jednak dominowali mieszkańcy Europy i Stanów
Zjednoczonych, co widać wyraźnie na rysunku \ref{fig:ch7_demo3} i w tabeli \ref{tab1:ch7_3} . Najwięcej
osób wypełniających formularz A pochodziło z Francji, Indii, Stanów Zjednoczonych i Wielkiej Brytanii. Z kolei w
przypadku formularza B, największe reprezentacje miały Stany Zjednoczone, Wielka Brytania i Niemcy.

\begin{table}[h!]
    \begin{center}
        \begin{tabular}{|l|r|r|}
            \hline
            Kraj pochodzenia  & Liczba osób & Procent całości \\
            \hline
            Antigua i Barbuda & 1           & 2,94\%          \\
            Argentyna         & 1           & 2,94\%          \\
            Chiny             & 1           & 2,94\%          \\
            Filipiny          & 1           & 2,94\%          \\
            Francja           & 3           & 8,82\%          \\
            Holandia          & 2           & 5,88\%          \\
            Indie             & 3           & 8,82\%          \\
            Niemcy            & 3           & 8,82\%          \\
            Nowa Zelandia     & 1           & 2,94\%          \\
            Polska            & 2           & 5,88\%          \\
            Portugalia        & 1           & 2,94\%          \\
            Rosja             & 1           & 2,94\%          \\
            Stany Zjednoczone & 7           & 20,59\%         \\
            Szwecja           & 1           & 2,94\%          \\
            Ukraina           & 1           & 2,94\%          \\
            Wielka Brytania   & 5           & 14,71\%         \\
            \hline
        \end{tabular}
    \end{center}
    \caption{Uczestnicy badania ze względu na kraj pochodzenia}\label{tab1:ch7_3}
\end{table}

\begin{figure}[h!]
    \centering
    \includegraphics[width=0.9\textwidth]{demo3.png}
    \caption{Kraj pochodzenia uczestników badania z podziałem na formularz A i B}
    \label{fig:ch7_demo3}
\end{figure}

\newpage

Interesującym aspektem badania był wiek, w którym uczestnicy rozpoczęli przygodę z grami wideo. Jak
pokazują dane w tabeli \ref{tab1:ch7_4}  i na rysunku \ref{fig:ch7_demo4} , zdecydowana większość osób
zaczęła grać w wieku 7-12 lat. Pozostali zadeklarowali, że zaczęli grać przed 7. rokiem życia
lub w okresie nastoletnim (13-19 lat). Jedna osoba rozpoczęła przygodę z grami po dwudziestym roku życia.

\begin{table}[h!]
    \begin{center}
        \begin{tabular}{|l|r|r|}
            \hline
            Wiek rozpoczęcia grania w gry wideo & Liczba osób & Procent całości \\
            \hline
            <7                                  & 7           & 20,59\%         \\
            7-12                                & 19          & 55,88\%         \\
            13-15                               & 5           & 14,71\%         \\
            16-19                               & 2           & 5,88\%          \\
            20+                                 & 1           & 2,94\%          \\
            \hline
        \end{tabular}
    \end{center}
    \caption{Uczestnicy badania ze względu na wiek rozpoczęcia grania w gry wideo}\label{tab1:ch7_4}
\end{table}

\begin{figure}[h!]
    \centering
    \includegraphics[width=0.9\textwidth]{demo4.png}
    \caption{Wiek rozpoczęcia grania w gry wideo przez uczestników badania z podziałem na formularz A i B}
    \label{fig:ch7_demo4}
\end{figure}

\newpage

Wreszcie, z danych przedstawionych na rysunku \ref{fig:ch7_demo5} i w tabeli \ref{tab1:ch7_5} wynika, że uczestnicy średnio nie spędzali
więcej niż 10 godzin tygodniowo na graniu. Najwięcej osób deklarowało granie przez mniej niż 5 godzin
tygodniowo lub w przedziale 5-10 godzin. Nieliczni przyznali, że grają ponad 20 godzin tygodniowo.

\begin{table}[h!]
    \begin{center}
        \begin{tabular}{|m{15em}|r|r|}
            \hline
            Średnia liczba godzin tygodniowo \newline przeznaczona na gry wideo & Liczba osób & Procent całości \\
            \hline
            <5h                                                                 & 16          & 47,06\%         \\
            5-10h                                                               & 10          & 29,41\%         \\
            10-15h                                                              & 2           & 5,88\%          \\
            15-20h                                                              & 2           & 5,88\%          \\
            >20h                                                                & 4           & 11,76\%         \\
            \hline
        \end{tabular}
    \end{center}
    \caption{Uczestnicy badania ze względu na średnią liczbę godzin tygodniowo przeznaczonych na gry wideo}\label{tab1:ch7_5}
\end{table}

\begin{figure}[h!]
    \centering
    \includegraphics[width=0.9\textwidth]{demo5.png}
    \caption{Średnia liczba godzin tygodniowo przeznaczona na gry wideo przez uczestników badania z podziałem na formularz A i B}
    \label{fig:ch7_demo5}
\end{figure}

\newpage

\section{Analiza danych}\label{section:ch7_2}

W ramach analizy zebranych danych zdecydowano się na wykorzystanie testów statystycznych do porównania różnic między
odpowiedziami zarówno w ramach jednego formularza (pomiędzy wersją bez AI i wersją z AI dla tego samego uczestnika)
jak i pomiędzy formularzami (porównanie odpowiedzi dla odpowiednich wersji pomiędzy formularzem A i B). Poziom
istotności statystycznej, z którym porównane są wartości prawdopodobieństwa testowego, został przyjęty na poziomie 0,05.
Pytanie, które osiągnęło w dowolnej rubryce wartość \textit{p value} poniżej poziomu istotności statystycznej,
oznaczone zostało w tabeli poprzez znak "*" a powodująca to wartość \textit{p value} została pogrubiona.

Ze względu na niespełnienie założenia normalności rozkładu przez wiele pozycji w teście Shapiro-Wilka (co zostało
przedstawione w tabelach \ref{tab1:ch7_10} oraz \ref{tab1:ch7_11}), nie można było zastosować testu t-Studenta,
pomimo spełnienia założenia równości wariancji w teście Levene'a (co widać w tabeli \ref{tab1:ch7_12}). Zdecydowano
się więc na użycie nieparametrycznych testów statystycznych.

\newpage

\begin{table}[!h]
    \begin{center}
        \begin{tabular}{|m{10em}|m{5em}|m{5em}|m{5em}|m{5em}|}
            \hline
            Pytanie                                                                     & Wartość statystyki (non-AI) & P-value (non-AI) & Wartość statystyki (AI) & P-value (AI)   \\
            \hline
            1. Tracę poczucie czasu                                                     & 0,911                       & 0,164            & 0,900                   & 0,113          \\
            2. Byłem/-am \newline zainteresowany/-a fabułą gry                          & 0,923                       & 0,243            & 0,881                   & 0,060          \\
            3. Czuję się inaczej\textbf{*}                                              & 0,855                       & \textbf{0,026}   & 0,886                   & 0,070          \\
            4. Czułem/-am, że mogę odkrywać różne rzeczy\textbf{*}                      & 0,755                       & \textbf{0,001}   & 0,837                   & \textbf{0,015} \\
            5. Gra wydaje się prawdziwa\textbf{*}                                       & 0,841                       & \textbf{0,017}   & 0,903                   & 0,127          \\
            6. Byłem/-am \newline w pełni zajęty/-a grą\textbf{*}                       & 0,850                       & \textbf{0,023}   & 0,896                   & 0,100          \\
            7. Denerwuję się\textbf{*}                                                  & 0,900                       & 0,111            & 0,814                   & \textbf{0,007} \\
            8. Czas jakby stanął w miejscu lub się zatrzymał\textbf{*}                  & 0,834                       & \textbf{0,013}   & 0,837                   & \textbf{0,015} \\
            9. Czuję się \newline rozkojarzony/-a\textbf{*}                             & 0,864                       & \textbf{0,034}   & 0,814                   & \textbf{0,007} \\
            10. Byłem/-am głęboko \newline skoncentrowany/-a \newline na grze\textbf{*} & 0,914                       & 0,181            & 0,848                   & \textbf{0,021} \\
            11. Zmęczyłem/-am się\textbf{*}                                             & 0,867                       & \textbf{0,038}   & 0,909                   & 0,152          \\
            12. Granie wydaje się automatyczne\textbf{*}                                & 0,924                       & 0,253            & 0,842                   & \textbf{0,017} \\
            13. Moje myśli \newline biegną szybko\textbf{*}                             & 0,932                       & 0,326            & 0,836                   & \textbf{0,014} \\
            14. Podobało mi się\textbf{*}                                               & 0,878                       & 0,054            & 0,875                   & \textbf{0,049} \\
            15. Gram bez zastanawiania się jak grać\textbf{*}                           & 0,805                       & \textbf{0,006}   & 0,896                   & 0,100          \\
            16. Granie sprawia, \newline że czuję się spokojny/-a\textbf{*}             & 0,911                       & 0,164            & 0,844                   & \textbf{0,018} \\
            17. Gram dłużej \newline niż zamierzałem/-am\textbf{*}                      & 0,820                       & \textbf{0,009}   & 0,855                   & \textbf{0,026} \\
            18. Naprawdę wczuwam się w grę                                              & 0,923                       & 0,246            & 0,885                   & 0,069          \\
            19. Czuję, że nie mogę przestać grać\textbf{*}                              & 0,786                       & \textbf{0,003}   & 0,744                   & \textbf{0,001} \\
            \hline
        \end{tabular}
    \end{center}
    \caption{Wyniki testu Shapiro-Wilka dla formularza A i obu wersji gier}\label{tab1:ch7_10}
\end{table}

\begin{table}[!h]
    \begin{center}
        \begin{tabular}{|m{10em}|m{5em}|m{5em}|m{5em}|m{5em}|}
            \hline
            Pytanie                                                                     & Wartość statystyki (non-AI) & P-value (non-AI) & Wartość statystyki (AI) & P-value (AI)   \\
            \hline
            1. Tracę poczucie czasu\textbf{*}                                           & 0,892                       & \textbf{0,029}   & 0,836                   & \textbf{0,003} \\
            2. Byłem/-am \newline zainteresowany/-a fabułą gry\textbf{*}                & 0,875                       & \textbf{0,014}   & 0,864                   & \textbf{0,009} \\
            3. Czuję się inaczej\textbf{*}                                              & 0,793                       & \textbf{0,001}   & 0,882                   & \textbf{0,019} \\
            4. Czułem/-am, że mogę odkrywać różne rzeczy\textbf{*}                      & 0,856                       & \textbf{0,007}   & 0,728                   & \textbf{0,000} \\
            5. Gra wydaje się prawdziwa\textbf{*}                                       & 0,827                       & \textbf{0,002}   & 0,867                   & \textbf{0,010} \\
            6. Byłem/-am \newline w pełni zajęty/-a grą\textbf{*}                       & 0,854                       & \textbf{0,006}   & 0,842                   & \textbf{0,004} \\
            7. Denerwuję się\textbf{*}                                                  & 0,792                       & \textbf{0,001}   & 0,847                   & \textbf{0,005} \\
            8. Czas jakby stanął w miejscu lub się zatrzymał\textbf{*}                  & 0,827                       & \textbf{0,002}   & 0,864                   & \textbf{0,009} \\
            9. Czuję się \newline rozkojarzony/-a\textbf{*}                             & 0,865                       & \textbf{0,010}   & 0,917                   & \textbf{0,088} \\
            10. Byłem/-am głęboko \newline skoncentrowany/-a \newline na grze\textbf{*} & 0,883                       & \textbf{0,020}   & 0,792                   & \textbf{0,001} \\
            11. Zmęczyłem/-am się\textbf{*}                                             & 0,860                       & \textbf{0,008}   & 0,902                   & \textbf{0,045} \\
            12. Granie wydaje się automatyczne\textbf{*}                                & 0,789                       & \textbf{0,001}   & 0,893                   & \textbf{0,031} \\
            14. Podobało mi się\textbf{*}                                               & 0,890                       & \textbf{0,000}   & 0,865                   & \textbf{0,010} \\
            15. Gram bez zastanawiania się jak grać\textbf{*}                           & 0,746                       & \textbf{0,006}   & 0,896                   & \textbf{0,034} \\
            16. Granie sprawia, \newline że czuję się spokojny/-a\textbf{*}             & 0,851                       & \textbf{0,006}   & 0,825                   & \textbf{0,002} \\
            17. Gram dłużej \newline niż zamierzałem/-am\textbf{*}                      & 0,840                       & \textbf{0,004}   & 0,860                   & \textbf{0,008} \\
            18. Naprawdę wczuwam się w grę\textbf{*}                                    & 0,876                       & \textbf{0,015}   & 0,875                   & \textbf{0,015} \\
            19. Czuję, że nie mogę przestać grać\textbf{*}                              & 0,850                       & \textbf{0,005}   & 0,882                   & \textbf{0,019} \\
            \hline
        \end{tabular}
    \end{center}
    \caption{Wyniki testu Shapiro-Wilka dla formularza B i obu wersji gier}\label{tab1:ch7_11}
\end{table}

\begin{table}[!h]
    \begin{center}
        \begin{tabular}{|m{10em}|m{5em}|m{5em}|m{5em}|m{5em}|}
            \hline
            Pytanie                                                           & Wartość statystyki (non-AI) & P-value (non-AI) & Wartość statystyki (AI) & P-value (AI) \\
            \hline
            1. Tracę poczucie czasu                                           & 0,613                       & 0,439            & 0,313                   & 0,580        \\
            2. Byłem/-am \newline zainteresowany/-a fabułą gry                & 0,121                       & 0,730            & 1,395                   & 0,246        \\
            3. Czuję się inaczej                                              & 1,202                       & 0,281            & 0,141                   & 0,710        \\
            4. Czułem/-am, że mogę odkrywać różne rzeczy                      & 3,149                       & 0,086            & 0,004                   & 0,949        \\
            5. Gra wydaje się prawdziwa                                       & 2,329                       & 0,137            & 0,054                   & 0,818        \\
            6. Byłem/-am \newline w pełni zajęty/-a grą                       & 0,070                       & 0,793            & 0,281                   & 0,600        \\
            7. Denerwuję się                                                  & 0,388                       & 0,538            & 0,168                   & 0,685        \\
            8. Czas jakby stanął w miejscu lub się zatrzymał                  & 1,340                       & 0,256            & 0,121                   & 0,731        \\
            9. Czuję się \newline rozkojarzony/-a                             & 0,980                       & 0,330            & 0,000                   & 1,000        \\
            10. Byłem/-am głęboko \newline skoncentrowany/-a \newline na grze & 0,009                       & 0,926            & 1,252                   & 0,272        \\
            11. Zmęczyłem/-am się                                             & 0,003                       & 0,956            & 0,059                   & 0,810        \\
            12. Granie wydaje się automatyczne                                & 0,408                       & 0,528            & 2,819                   & 0,103        \\
            13. Moje myśli \newline biegną szybko                             & 0,693                       & 0,411            & 0,604                   & 0,443        \\
            14. Podobało mi się                                               & 0,163                       & 0,689            & 2,875                   & 0,100        \\
            15. Gram bez zastanawiania się jak grać                           & 1,740                       & 0,196            & 0,227                   & 0,637        \\
            16. Granie sprawia, \newline że czuję się spokojny/-a             & 0,000                       & 1,000            & 1,026                   & 0,319        \\
            17. Gram dłużej \newline niż zamierzałem/-am                      & 1,355                       & 0,253            & 0,320                   & 0,576        \\
            18. Naprawdę wczuwam się w grę                                    & 0,254                       & 0,617            & 0,399                   & 0,532        \\
            19. Czuję, że nie mogę przestać grać                              & 0,508                       & 0,481            & 0,640                   & 0,430        \\
            \hline
        \end{tabular}
    \end{center}
    \caption{Wyniki testu Levene'a porównującego wariancję pomiędzy formularzem A i B odpowiednio dla wersji gier bez AI i z AI}\label{tab1:ch7_12}
\end{table}

\clearpage

W celu porównania różnic pomiędzy wersją bez AI i wersją z AI wewnątrz formularzy A i B, wykorzystano test
nieparametryczny Wilcoxona z zaztosowaną korektą na ciągłość. Analizując wyniki dla formularza A, uzyskano istotne statystycznie różnice dla
pozycji "Granie wydaje się automatyczne" (p = 0,031), "Gra wydaje się prawdziwa" (p = 0,012) oraz "Czułem/-am, że
mogę odkrywać różne rzeczy" (p = 0,003). W przypadku formularza B, istotne różnice odnotowano dla pozycji
"Czułem/-am, że mogę odkrywać różne rzeczy" (p = 0,002), "Granie wydaje się automatyczne" (p = 0,024),
"Gram bez zastanawiania się jak grać" (p = 0,008) oraz "Gram dłużej niż zamierzałem/-am" (p = 0,030). Pozycje 1 i 7
były stosunkowo blisko progu.

\begin{table}[!h]
    \begin{center}
        \begin{tabular}{|m{12em}|m{5em}|m{4em}|m{5em}|m{4em}|}
            \hline
            Pytanie                                                           & Wartość statystyki (A) & P-value (A)    & Wartość statystyki (B) & P-value (B)    \\
            \hline
            1. Tracę poczucie czasu                                           & 8                      & 0,152          & 12,5                   & 0,067          \\
            2. Byłem/-am \newline zainteresowany/-a fabułą gry                & 6                      & 0,783          & 27,5                   & 0,197          \\
            3. Czuję się inaczej                                              & 4                      & 0,408          & 21                     & 0,533          \\
            4. Czułem/-am, że mogę odkrywać różne rzeczy\textbf{*}            & 1,5                    & \textbf{0,003} & 0                      & \textbf{0,002} \\
            5. Gra wydaje się prawdziwa\textbf{*}                             & 0                      & \textbf{0,012} & 19,5                   & 0,759          \\
            6. Byłem/-am \newline w pełni zajęty/-a grą                       & 6,5                    & 0,890          & 41,5                   & 0,803          \\
            7. Denerwuję się                                                  & 10,5                   & 0,605          & 7                      & 0,066          \\
            8. Czas jakby stanął w miejscu lub się zatrzymał                  & 11                     & 0,665          & 25                     & 0,266          \\
            9. Czuję się \newline rozkojarzony/-a                             & 1,5                    & 0,586          & 22,5                   & 0,193          \\
            10. Byłem/-am głęboko \newline skoncentrowany/-a \newline na grze & 6,5                    & 0,890          & 22,5                   & 0,636          \\
            11. Zmęczyłem/-am się                                             & 18                     & 0,624          & 28                     & 0,684          \\
            12. Granie wydaje się automatyczne\textbf{*}                      & 1                      & \textbf{0,031} & 5                      & \textbf{0,024} \\
            13. Moje myśli \newline biegną szybko                             & 0                      & 0,089          & 20                     & 0,437          \\
            14. Podobało mi się                                               & 9                      & 0,830          & 20                     & 0,236          \\
            15. Gram bez zastanawiania się jak grać\textbf{*}                 & 5                      & 0,077          & 3                      & \textbf{0,008} \\
            16. Granie sprawia, \newline że czuję się spokojny/-a             & 14,5                   & 0,669          & 34,5                   & 0,745          \\
            17. Gram dłużej \newline niż zamierzałem/-am\textbf{*}            & 6                      & 0,766          & 6                      & \textbf{0,030} \\
            18. Naprawdę wczuwam się w grę                                    & 10,5                   & 1,000          & 16                     & 0,243          \\
            19. Czuję, że nie mogę przestać grać                              & 2                      & 0,773          & 21                     & 0,524          \\
            \hline
        \end{tabular}
    \end{center}
    \caption{Wyniki testu Wilcoxona porównującego zmianę ocen pytań wewnątrz formularza A i B}\label{tab1:ch7_13}
\end{table}

\newpage

Wykonano porównanie odpowiednich wariantów gry (bez AI i z AI) pomiędzy formularzem A i B za pomocą testu
Manna-Whitneya z zastosowaną korektą na ciągłość. Stwierdzono istotne statystycznie różnice dla pozycji "Czułem/-am, że mogę odkrywać różne rzeczy"
zarówno w wersji bez AI (p = 0,004), jak i w wersji z AI (p = 0,004). Różnice pomiędzy formularzami
A i B dla wersji bez AI zaobserwowano również dla pozycji "Gra wydaje się prawdziwa" (p = 0,023) oraz "Granie wydaje
się automatyczne" (p = 0,044). W przypadku wersji z użyciem AI, pomiędzy formularzami A i B odnotowano różnice dla
pozycji "Tracę poczucie czasu" (p = 0,025), "Granie sprawia, że czuję się spokojny/-a" (p = 0,015) oraz "Gram
dłużej niż zamierzałem/-am" (p = 0,048). Pozycje 9 i 19 były stosunkowo blisko progu.

\begin{table}[h!]
    \begin{center}
        \begin{tabular}{|m{10em}|m{5em}|m{5em}|m{5em}|m{5em}|}
            \hline
            Pytanie                                                           & Wartość statystyki (non-AI) & P-value (non-AI) & Wartość statystyki (AI) & P-value (AI)   \\
            \hline
            1. Tracę poczucie czasu\textbf{*}                                 & 133                         & 0,816            & 77                      & \textbf{0,025} \\
            2. Byłem/-am \newline zainteresowany/-a fabułą gry                & 118                         & 0,440            & 100                     & 0,154          \\
            3. Czuję się inaczej                                              & 121                         & 0,503            & 123,5                   & 0,567          \\
            4. Czułem/-am, że mogę odkrywać różne rzeczy\textbf{*}            & 58,5                        & \textbf{0,004}   & 221                     & \textbf{0,004} \\
            5. Gra wydaje się prawdziwa\textbf{*}                             & 77                          & \textbf{0,023}   & 133,5                   & 0,829          \\
            6. Byłem/-am \newline w pełni zajęty/-a grą                       & 112                         & 0,323            & 101                     & 0,167          \\
            7. Denerwuję się                                                  & 154,5                       & 0,611            & 138,5                   & 0,971          \\
            8. Czas jakby stanął w miejscu lub się zatrzymał                  & 106                         & 0,227            & 95                      & 0,110          \\
            9. Czuję się \newline rozkojarzony/-a                             & 115,5                       & 0,389            & 85                      & 0,051          \\
            10. Byłem/-am głęboko \newline skoncentrowany/-a \newline na grze & 106,5                       & 0,237            & 107                     & 0,242          \\
            11. Zmęczyłem/-am się                                             & 133,5                       & 0,829            & 159                     & 0,504          \\
            12. Granie wydaje się automatyczne\textbf{*}                      & 84,5                        & \textbf{0,044}   & 88                      & 0,064          \\
            13. Moje myśli \newline biegną szybko                             & 120,5                       & 0,493            & 92,5                    & 0,090          \\
            14. Podobało mi się                                               & 123                         & 0,552            & 106                     & 0,221          \\
            15. Gram bez zastanawiania się jak grać                           & 107                         & 0,220            & 113,5                   & 0,352          \\
            16. Granie sprawia, \newline że czuję się spokojny/-a\textbf{*}   & 97,5                        & 0,129            & 72,5                    & \textbf{0,015} \\
            17. Gram dłużej \newline niż zamierzałem/-am\textbf{*}            & 129,5                       & 0,719            & 84,5                    & \textbf{0,048} \\
            18. Naprawdę wczuwam się w grę                                    & 119,5                       & 0,474            & 105,5                   & 0,222          \\
            19. Czuję, że nie mogę przestać grać                              & 102                         & 0,175            & 87,5                    & 0,062          \\
            \hline
        \end{tabular}
    \end{center}
    \caption{Wyniki testu Manna-Whitneya porównującego skalę ocen wersji gier bez AI i z AI pomiędzy formularzem A i B}\label{tab1:ch7_14}
\end{table}

\newpage

Test Wilcoxona wykazał istotne różnice pomiędzy wersjami gry bez AI i z AI w obrębie formularzy A i B
dla pozycji związanej z odkrywaniem świata gry oraz pozycji związanej z automatycznością. Przejście z
wersji bez AI na wersję z AI w ramach formularza A wykazało istotną różnicę dotyczącą postrzegania
prawdziwości świata gry. W przypadku formularza B "zabranie" AI związane było z istotnymi zmianami
w obrębie postrzegania czasu poświęconego na grę oraz gry bez zastanawiania się.

Analiza testem Manna-Whitneya ujawniła różnice pomiędzy formularzami A i B dla niektórych pozycji, przy 
czym część różnic była widoczna tylko w wersji bez AI, a część tylko w wersji z AI. Sugeruje to, że formuła 
eksperymentu (ogrywanie dwóch wersji jedna po drugiej) miała wpływ na postrzeganie niektórych aspektów
rozgrywki. Może to oznaczać, że pierwsze doświadczenie jest de facto oceną bazową gry a drugie wypełnienie
kwestionariusza ujawnia wyraźne różnice po dodaniu lub zabraniu AI.

Dla niektórych pozycji (np. 1, 7, 9, 19) wyniki były bliskie progu istotności statystycznej, co 
sugeruje możliwość uzyskania dodatkowych różnic przy większej liczbie uczestników badania.

\vspace{10pt}

Oto najczęściej pojawiające się motywy w ramach pytań otwartych dla wersji bez AI:

\begin{enumerate}
    \item Odczucie braku realizmu i statyczności interakcji z NPC.
    \item Przeciętne zainteresowanie rozmową z NPC, gdyż gracze nie czuli się naprawdę zaangażowani.
    \item Niezadowolenie z ograniczonej ilości informacji otrzymywanych od NPC.
    \item Duży podział jeśli chodzi o ostateczną przyjemność interakcji: część graczy
        narzekała na brak interakcji a dla części predefiniowane dialogi były na tyle dobre,
        że nie przeszkadzało im to.
\end{enumerate}

Natomiast dla wersji z AI wystąpiły najczęściej takie motywy:

\begin{enumerate}
    \item Większe zainteresowanie rozmową z NPC i możliwością zadawania dowolnych pytań.
    \item Odczucie większego realizmu i lepszej immersji podczas interakcji.
    \item Zadowolenie z możliwości uzyskania większej ilości informacji od NPC.
    \item Frustracja spowodowana czasem odpowiedzi NPC lub ich niekompletnymi odpowiedziami.
    \item Rozproszenie uwagi od głównej rozgrywki z powodu zbyt wielu opcji konwersacji.
    \item Większość zadeklarowała duże poczucie przyjemności z interakcji z NPC. 
\end{enumerate}

Dokładne odpowiedzi na pytania otwarte zostały zawarte w ramach dodatku \ref{appendix:C}.