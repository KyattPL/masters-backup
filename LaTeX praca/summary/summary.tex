% !TEX encoding = UTF-8 Unicode 
% !TEX root = praca.tex

\chapter*{Podsumowanie}

Przeprowadzone badania dostarczyły istotnych wyników dotyczących wpływu rozwiązania opartego na \gls{ai} na odbiór
interakcji z bohaterami niezależnymi (\gls{npc}) w grze wideo. Testy statystyczne Wilcoxona i Manna-Whitneya wykazały
znaczące różnice w postrzeganiu rozgrywki przez graczy pomiędzy wersjami z \gls{ai} i bez \gls{ai}.

W wersji z \gls{ai} gracze odczuwali większy realizm i imersję podczas interakcji z \gls{npc}, a także możliwość uzyskania
większej ilości informacji. Jednocześnie część graczy była sfrustrowana czasem odpowiedzi \gls{npc} lub ich niekompletnymi
wypowiedziami. W wersji bez \gls{ai} gracze narzekali na brak realizmu, statyczność interakcji oraz ograniczoną ilość
informacji otrzymywanych od \gls{npc}.

Wyniki analiz jakościowych z pytań otwartych potwierdziły powyższe obserwacje, wskazując na większe zainteresowanie
rozmowami z \gls{npc} i wyższy poziom satysfakcji w wersji z \gls{ai}. Jednocześnie część graczy odczuwała rozproszenie uwagi
od głównej rozgrywki z powodu dużej ilości opcji konwersacji.

Aby rozwinąć badania i uzyskać bardziej miarodajne rezultaty, zaleca się następujące kroki:

\begin{itemize}
    \item Replikacja badań na większą skalę z większą liczbą uczestników, co pozwoli na uzyskanie bardziej
          wiarygodnych wyników statystycznych.
    \item Eksploracja innych gatunków gier poza badanym, aby sprawdzić jak system \gls{ai} wpływa na odbiór interakcji
          z \gls{npc} w różnych kontekstach.
    \item Przetestowanie alternatywnych dużych modeli językowych, które mogą dostarczyć innego rodzaju odpowiedzi
          \gls{npc} i wpłynąć na doświadczenie graczy.
    \item Ponowne badania w przyszłości po pojawieniu się nowszych, ulepszonych modeli językowych, które mogą
          podnieść jakość interakcji z \gls{npc} na wyższy poziom.
\end{itemize}

Niniejsze badania stanowią ważny krok w kontekście eksploracji potencjału systemów opartych na dużych językach modelowych do
zwiększenia realizmu i imersji w grach wideo. Dalsze prace pozwolą na pełniejsze zrozumienie wpływu tych 
systemów na doświadczenie graczy.